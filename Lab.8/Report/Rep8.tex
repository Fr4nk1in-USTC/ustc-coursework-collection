% tempalte.tex
% Author: 傅申
% 本文件是 实验报告模板.docx 与 实验报告模板.tex 的衍生作品,其著作权属于中国科学技术大学计算机实验教学中心.

% 本文件可使用TeXLive发行版(作者使用的是TeXLive 2021,但理论上2016及以后均可)的xelatex命令编译.
\documentclass[UTF8,fontset=fandol]{ctexart}
\usepackage{LabReport}
\usetikzlibrary{arrows,shapes,chains}
\begin{document}
\maketitle{信号处理及有限状态机}
\section*{实验目的}
%列举本实验的实验目的,除指导书上列出的之外,鼓励自行总结及扩展。
\begin{itemize}
    \item 进一步熟悉 FPGA 开发的整体流程
    \item 掌握几种常见的信号处理技巧
    \item 掌握有限状态机的设计方法
    \item 能够使用有限状态机设计功能电路
\end{itemize}
\section*{实验环境}
%列举本实验所用到的各种软硬件环境,如EDA工具、实验平台、实验设备等。
\begin{itemize}
    \item Windows PC
    \item Microsoft Visual Studio Code
    \item Logisim
    \item Xilinx Design Tools Vivado HL Design Edition 2019.1 
\end{itemize}
\section*{实验练习}
%如无特殊说明,则应完成对应实验手册上的所有练习题目,将过程和结果以图文并茂的形式体现在本报告中,建议实验过程中随手保存各种截图。
\begin{ExQuestions}
  \question Step 5. 中的代码共有 4 个状态: 0, 1, 2, 3, 分别编码为 00, 01, 10, 11. 其状态跳转图如图 \ref{fig:state_transfer:1}.
  \begin{figure}[!htbp]
      \centering
      \begin{tikzpicture}[->,>=stealth', align = center, auto, node distance=3cm, node/.style = {draw, circle, fill = none, minimum size = 1cm}]

          \node[node](0){0\\(00)};
          \node[above of = 0, node distance = 1cm](4){输出0};
          \node[node, right of = 0](1){1\\(01)};
          \node[above of = 1, node distance = 1cm](5){输出0};
          \node[node, below of = 1](2){2\\(10)};
          \node[below of = 2, node distance = 1cm](6){输出0};
          \node[node, below of = 0](3){3\\(11)};
          \node[below of = 3, node distance = 1cm](7){输出1};
          \path[]
          (0) edge (1)
          (1) edge (2)
          (2) edge (3)
          (3) edge (0);
      \end{tikzpicture}
      \caption{Step 5. 的状态跳转图}
      \label{fig:state_transfer:1}    
  \end{figure}

  只有当状态为 3 时才输出 1, 其他状态都输出 0. 由此写出有限状态机的代码如下
  \begin{lstlisting}[style = verilogstyle, caption = {Step 5. 中代码改写为有限状态机的形式}]
module test(
    input clk, rst,
    output led
);
parameter STATE_0 = 2'h0;
parameter STATE_1 = 2'h1;
parameter STATE_2 = 2'h2;
parameter STATE_3 = 2'h3;
reg [1:0] curr_state, next_state;
// FSM Part 1
always @(*) case (curr_state)
    STATE_0: next_state = STATE_1;
    STATE_1: next_state = STATE_2;
    STATE_2: next_state = STATE_3;
    STATE_3: next_state = STATE_0;
    default: next_state = STATE_0;
endcase
// FSM Part 2
always @(posedge clk or posedge rst) begin
    if (rst) curr_state <= STATE_0;
    else curr_state <= next_state;
end
// FSM Part 3
assign led = (curr_state == STATE_3);
endmodule
  \end{lstlisting}
  \question 首先, 采用 Step 2. 中的方式取 sw 信号的边缘, 将最后的与门换成同或门, 即可同时取到 sw 的上升沿与下降沿.
  然后设计一个位宽为 4 bit 的 D 触发器, 其复位值为 \texttt{4'b0000}. 将 sw 的信号边缘连接到 D 触发器的时钟输入, 
  并且将计数器电路连接到 D 输入. 最后得到的电路如图 \ref{fig:circuit:3}. 电路设计过程如图 \ref{fig:circuit}.
  \begin{figure}[!htbp]
      \centering
      \subimg{取 sw 信号边缘}{.8}{images/fig1.1.jpg}{fig:circuit:1}\\
      \subimg{位宽为 4 bit 的 D 触发器}{.8}{images/fig1.2.jpg}{fig:circuit:2}\\
      \subimg{最终电路}{.8}{images/fig1.3.jpg}{fig:circuit:3}
      \caption{计数器电路}
      \label{fig:circuit}
  \end{figure}
  \question 因为要根据按钮按下的瞬间进行计数, 所以首先需要取 \texttt{btn} 信号的边缘. 取信号边缘的模块如下
  \begin{lstlisting}[style = verilogstyle, caption = {取信号边缘的模块}, label = {Code:2}]
// ./Prob.3/Prob.3.srcs/sources_1/new/SignalEdge.v
module signal_edge(
    input clk,
    input signal,
    output signal_edge
    );
reg signal_r1, signal_r2;
always @(posedge clk) signal_r1 <= signal;
always @(posedge clk) signal_r2 <= signal_r1;
assign signal_edge = signal_r1 & (~signal_r2);
endmodule
  \end{lstlisting}
  在 \texttt{btn\_edge} 信号有效时根据 \texttt{sw} 的情况进行计数.
  因为需要用到 2 个数码管, 所以我们需要生成一个 100 Hz 的时钟信号, 并采用时分复用的方式进行显示.
  最后的 Verilog 代码如代码 \ref{Code:3}. 
  \begin{lstlisting}[style = verilogstyle, caption = {题目 3. 的 Verilog 代码}, label = {Code:3}]
// ./Prob.3/Prob.3.srcs/sources_1/new/Counter.v
module counter(
    input clk_100mhz,
    input sw,
    input btn,
    input rst,
    output reg hexplay_an,
    output reg [3:0] hexplay_data
    );
// generate a 100 hz clock
wire clk_100hz;
reg [19:0] clk_cnt;
assign clk_100hz = (clk_cnt >= 500000);
always @(posedge clk_100mhz)
begin
    if (clk_cnt >= 1000000) clk_cnt = 0;
    else clk_cnt = clk_cnt + 20'h00001;
end
// get the edge of btn signal
wire btn_edge;
signal_edge getBtnEdge(.clk(clk_100mhz), 
                       .signal(btn), 
                       .signal_edge(btn_edge));
// counter
reg [7:0] cnt;
always @(posedge clk_100mhz)
begin
    if (rst) cnt <= 8'h1f;
    else if (btn_edge)
    begin
        if (sw)
        begin
            if (cnt >= 8'hff) cnt <= 8'h00;
            else cnt <= cnt + 8'h01;
        end
        else
        begin
            if (cnt == 8'h00) cnt <= 8'hff;
            else cnt <= cnt - 8'h01;
        end
    end
end
// segplay
always @(posedge clk_100mhz)
begin
    if (clk_100hz) 
    begin
        hexplay_an = 1'b1;
        hexplay_data = cnt[7:4];
    end
    else
    begin
        hexplay_an = 1'b0;
        hexplay_data = cnt[3:0];
    end
end
endmodule
  \end{lstlisting}
  部分 xdc 约束文件代码如下
  \begin{lstlisting}[basicstyle=\footnotesize\ttfamily, numbers = left, language = XML, frame=lrtb, frameround=tttt, caption={题目 3. 的 xdc 文件}, breaklines=true]
## Clock signal
#IO_L12P_T1_MRCC_35 Sch=clk100mhz
set_property -dict { PACKAGE_PIN E3    IOSTANDARD LVCMOS33 } [get_ports { clk_100mhz }]; 

## FPGAOL SWITCH

set_property -dict { PACKAGE_PIN D14   IOSTANDARD LVCMOS33 } [get_ports { sw }];
set_property -dict { PACKAGE_PIN F16   IOSTANDARD LVCMOS33 } [get_ports { rst }];

## FPGAOL HEXPLAY

set_property -dict { PACKAGE_PIN A14   IOSTANDARD LVCMOS33 } [get_ports { hexplay_data[0] }];
set_property -dict { PACKAGE_PIN A13   IOSTANDARD LVCMOS33 } [get_ports { hexplay_data[1] }];
set_property -dict { PACKAGE_PIN A16   IOSTANDARD LVCMOS33 } [get_ports { hexplay_data[2] }];
set_property -dict { PACKAGE_PIN A15   IOSTANDARD LVCMOS33 } [get_ports { hexplay_data[3] }];
set_property -dict { PACKAGE_PIN B17   IOSTANDARD LVCMOS33 } [get_ports { hexplay_an }];

## FPGAOL BUTTON & SOFT_CLOCK

set_property -dict { PACKAGE_PIN B18   IOSTANDARD LVCMOS33 } [get_ports { btn }];
  \end{lstlisting}
  最后生成比特流后烧写在 FPGAOL 平台, 效果如图 \ref{fig:fpga:1}.
  \img{.8}{images/fig.2.jpg}{题目 3. 在 FPGA上的显示效果}{fig:fpga:1}

  \question 由题意可知, 有限状态机有 4 个状态, 对应输入序列以 \texttt{0}, \texttt{1}, \texttt{11}, \texttt{110} 结尾, 分别编码为 00, 01, 10, 11. 只有在状态 11 且输入的 \texttt{sw} 为 0 时, 计数器才加 1. 状态转移图如图 \ref{fig:state_transfer:2}.
  \begin{figure}[!htbp]
    \centering
    \begin{tikzpicture}[->,>=stealth', align = center, auto, node distance=4cm, node/.style = {draw, circle, fill = none, minimum size = 1cm}]
      \node[node] (s0) {State\\00};
      \node[node, right of = s0] (s1) {State\\01};
      \node[node, below of = s1] (s2) {State\\10};
      \node[node, below of = s0] (s3) {State\\11};
      \node[above of= s0, node distance = 1.1cm] (s0_0) {计数器不变};
      \node[above of= s1, node distance = 1.1cm] (s1_1) {计数器不变};
      \node[below of= s2, node distance = 1.1cm] (s2_1) {计数器不变};
      \node[below of= s3, node distance = 1.4cm] (s3_0) {0: 计数器加 1 \\1: 计数器不变};
      \path[]
      (s0) edge [out = 150, in = 210, looseness = 4] node [left] {0} (s0)
      (s0) edge [bend left = 20] node [above] {1} (s1)
      (s1) edge [bend left = 20] node [below] {0} (s0)
      (s1) edge node[right] {1} (s2)
      (s2) edge [out = 30, in = 330, looseness = 4] node [right] {1} (s2)
      (s2) edge node [below] {0} (s3)
      (s3) edge node [left] {0,1} (s0);
    \end{tikzpicture}
    \caption{题目 4. 的状态转移图}
    \label{fig:state_transfer:2}
  \end{figure}
  由题意可知, 我们仍然需要取 \texttt{btn} 信号的边缘, 所以仍需要用到代码 \ref{Code:2} 的模块.
  而需要驱动 6 个数码管, 我们则需要 300 Hz 的时钟/脉冲信号, 但是 300 Hz 不好实现, 因此这里使用 400 Hz 的脉冲信号进行驱动.
  最后的 Verilog 模块代码如下
  \begin{lstlisting}[style = verilogstyle, caption = {题目 4. 的 Verilog 模块}, breaklines=true]  
module array_detect(
    input clk_100mhz,
    input btn,
    input sw,
    output reg [3:0] hexplay_data,
    output reg [2:0] hexplay_an
    );
// generate a 400 Hz pulse
wire pulse_400hz;
reg [18:0] pulse_cnt;

assign pulse_400hz = (pulse_cnt == 19'h00001);
always @(posedge clk_100mhz)
begin
    if (pulse_cnt >= 19'h3d090) pulse_cnt <= 19'h00000;
    else pulse_cnt <= pulse_cnt + 19'h00001;
end

// get btn signal edge
wire btn_edge;
signal_edge getBtnEdge(.clk(clk_100mhz),
                        .signal(btn),
                        .signal_edge(btn_edge));

// input array process
reg [3:0] input_array;
initial input_array = 4'h0;
always @(posedge clk_100mhz)
begin
        if (btn_edge) input_array <= {input_array[2:0], sw};
end
    
// FSM
parameter STATE_0 = 2'b00;
parameter STATE_1 = 2'b01;
parameter STATE_2 = 2'b10;
parameter STATE_3 = 2'b11;

reg [1:0] curr_state, next_state;
reg [3:0] cnt;
initial cnt <= 4'h0;
// Part 1
always @(*) 
begin
    if (sw) 
    begin
        case(curr_state)
            STATE_0: next_state = STATE_1;
            STATE_1: next_state = STATE_2;
            STATE_2: next_state = STATE_2;
            STATE_3: next_state = STATE_0;
            default: next_state = STATE_0;
        endcase
    end
    else
    begin
        case(curr_state)
            STATE_0: next_state = STATE_0;
            STATE_1: next_state = STATE_0;
            STATE_2: next_state = STATE_3;
            STATE_3: next_state = STATE_0;
            default: next_state = STATE_0;
        endcase
    end
end
// Part 2
always @(posedge clk_100mhz)
    if (btn_edge) curr_state <= next_state;
// Part 3
always @(posedge clk_100mhz)
begin
    if (btn_edge) 
    begin
        if (curr_state == STATE_3 && sw == 0)
        begin
            if (cnt >= 4'hf) cnt <= 4'h0;
            else cnt <= cnt + 4'h1;
        end 
    end
end

// Segplay
always @(posedge clk_100mhz)
begin
    if (pulse_400hz)
    begin
        if (hexplay_an >= 3'h7)
            hexplay_an <= 3'h0;
        else if (hexplay_an == 3'h0 ||
                    hexplay_an == 3'h5)
                hexplay_an <= hexplay_an + 3'h2;
        else hexplay_an <= hexplay_an + 3'h1;
    end
end
always @(posedge clk_100mhz)
begin
    case (hexplay_an)
        3'h0: hexplay_data <= cnt;
        3'h2: hexplay_data <= {3'b000, input_array[0]};
        3'h3: hexplay_data <= {3'b000, input_array[1]};
        3'h4: hexplay_data <= {3'b000, input_array[2]};
        3'h5: hexplay_data <= {3'b000, input_array[3]};
        3'h7: hexplay_data <= {2'b00, curr_state};
        default: hexplay_data <= 4'h0;
    endcase
end
endmodule
  \end{lstlisting}
  可以看出, 我们使用了两端的两个数码管以及中间的四个数码管, 从左到右依次为: 状态编码, 最近输入四位数值, 目标序列个数. 工程的 xdc 约束文件代码如下
  \begin{lstlisting}[basicstyle = \footnotesize\ttfamily, breaklines=true, numbers = left, language = XML, frame=lrtb, frameround=tttt, caption={题目 4. 的 xdc 文件}]
## Clock signal
#IO_L12P_T1_MRCC_35 Sch=clk100mhz
set_property -dict { PACKAGE_PIN E3    IOSTANDARD LVCMOS33 } [get_ports { clk_100mhz }]; 
## FPGAOL SWITCH

set_property -dict { PACKAGE_PIN D14   IOSTANDARD LVCMOS33 } [get_ports { sw }];

## FPGAOL HEXPLAY

set_property -dict { PACKAGE_PIN A14   IOSTANDARD LVCMOS33 } [get_ports { hexplay_data[0] }];
set_property -dict { PACKAGE_PIN A13   IOSTANDARD LVCMOS33 } [get_ports { hexplay_data[1] }];
set_property -dict { PACKAGE_PIN A16   IOSTANDARD LVCMOS33 } [get_ports { hexplay_data[2] }];
set_property -dict { PACKAGE_PIN A15   IOSTANDARD LVCMOS33 } [get_ports { hexplay_data[3] }];
set_property -dict { PACKAGE_PIN B17   IOSTANDARD LVCMOS33 } [get_ports { hexplay_an[0] }];
set_property -dict { PACKAGE_PIN B16   IOSTANDARD LVCMOS33 } [get_ports { hexplay_an[1] }];
set_property -dict { PACKAGE_PIN A18   IOSTANDARD LVCMOS33 } [get_ports { hexplay_an[2] }];

## FPGAOL BUTTON & SOFT_CLOCK

set_property -dict { PACKAGE_PIN B18   IOSTANDARD LVCMOS33 } [get_ports { btn }];
  \end{lstlisting}
  生成比特流后烧写如 FPGAOL 平台, 当输入序列为 ``\texttt{0011001110011}'' 时, 显示效果如图 \ref{fig:fpga:2}.
  \img{.8}{images/fig.3.jpg}{题目 4. 在 FPGA 上的显示效果}{fig:fpga:2}
\end{ExQuestions}
\section*{总结与思考}
%请填写对于本次实验的总结与思考,鼓励填写对于本实验或者本课程的各种建议和吐槽。
\begin{description}
    \item[收获] 学习了如何对输入信号进行处理, 能够构造简单的有限状态机.
    \item[难易程度] 较难
    \item[任务量] 中等
    \item[建议] 暂无
  \end{description}
\end{document}
