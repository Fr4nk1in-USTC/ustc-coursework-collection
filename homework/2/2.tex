\documentclass[boxes]{homework}

% This is a slightly-more-than-minimal document that uses the homework class.
% See the README at http://git.io/vZWL0 for complete documentation.

\name{傅申 PB20000051}        % Replace (Your Name) with your name.
\term{2022 秋}     % Replace (Current Term) with the current term.
\course{算法基础}    % Replace (Course Name) with the course name.
\hwnum{2}          % Replace (Number) with the number of the homework.
\hwname{作业}
\problemname{}
\solutionname{解:}

% Load any other packages you need here.
\usepackage[
    a4paper,
    top = 2.54cm,
    bottom = 2.54cm,
    left = 1.91cm,
    right = 1.91cm,
    includeheadfoot
]{geometry}
\fancyfootoffset{0pt} % make fancyhdr work properly
\usepackage{ctex}
\usepackage{tikz}

\allowdisplaybreaks[4]

\begin{document}
\problemchap{4}
\problempart{1}
\problemnumber{4}
\begin{problem}
假定修改最大子数组问题的定义, 允许结果为空子数组, 其和为 0. 你应该如何修改现有算
法, 使它们能允许空子数组为最终结果?
\end{problem}

\begin{solution}
    在 {\sc Find-Maximum-Subarray} 过程中, 如果需要返回的子数组和为负, 则改为返
    回空子数组.
\end{solution}

\begin{problem}
使用如下思想为最大子数组问题设计一个非递归的, 线性时间的算法. 从数组的左边开始,
从左至右处理, 记录到目前为止以及处理过的最大子数组. 若已知 $A[1..j]$ 的最大子数
组, 基于如下性质将解扩展为 $A[1..j + 1]$ 的最大子数组: $A[1..j + 1]$ 的最大子数
组要么是 $A[1..j]$ 的最大子数组, 要么是某个子数组 $A[i..j + 1] (1 \leqslant i
    \leqslant j + 1)$. 在已知 $A[1..j]$ 的最大子数组的情况下, 可以在线性时间内找
出形如 $A[i..j + 1]$ 的最大子数组.
\end{problem}

\begin{solution}
    如下

    \begin{algo}
        \caption{LINEAR-TIME-MAX-SUBARRAY(A)}
        \label{algo:linear-time-max-subarray}
        $max\text{-}sum = -\infty$\;
        $max\text{-}tail\text{-}sum = 0$\;
        $tail\text{-}left = 1$\;
        \For{$j = 1$ \textbf{to} $A.length$}{
            $max\text{-}tail\text{-}sum = max\text{-}tail\text{-}sum + A[j]$\;
            \If{$max\text{-}tail\text{-}sum > max\text{-}sum$}{
                $max\text{-}sum = max\text{-}tail\text{-}sum$\;
                $max\text{-}left = tail\text{-}left$\;
                $max\text{-}right = j$\;
            }
            \If{$max\text{-}tail\text{-}sum < 0$}{
                $max\text{-}tail\text{-}sum = 0$\;
                $tail\text{-}left = j + 1$\;
            }
        }
        \Return $max\text{-}left, max\text{-}right, max\text{-}sum$\;
    \end{algo}
\end{solution}

%%%% Problem 4.3-3 %%%%
\problempart{3}
\problemnumber{3}
\begin{problem}
我们看到 $T(n) = 2T \left( \left\lfloor \frac{ n }{ 2 } \right\rfloor\right) +
    n$ 的解为 $O(n\lg n)$. 证明 $\Omega(n\lg n)$ 也是这个递归式的解. 从而得出结
论: 解为 $\Theta(n\lg n)$.
\end{problem}

\begin{solution}
    恰当选择常数 $ \frac{ 1 }{ 3 } \geqslant c > 0$, 有
    $T(n) \geqslant cn\lg n$. 假定此上界对所有正数 $m < n$ 都成立, 特别对于
    $m = \left\lfloor \frac{n}{2} \right\rfloor$, 有
    $T \left( \left\lfloor \frac{ n }{ 2 } \right\rfloor\right) \geqslant
        c\left\lfloor \frac{ n }{ 2 } \right\rfloor \lg
        \left( \left\lfloor \frac{ n }{ 2 } \right\rfloor\right)$, 将此代入递归
    式, 有
    \begin{equation}
        \begin{aligned}
            T(n)
             & \geqslant 2 \left(
            c \left\lfloor \frac{ n }{ 2 } \right\rfloor \lg
            \left( \left\lfloor \frac{ n }{ 2 } \right\rfloor\right)
            \right) + n
            \geqslant c (n - 1) \lg \frac{ n - 1 }{ 2 } + n           \\
             & = c (n - 1)
            \left( \lg n - 1 - \lg \frac{ n }{ n - 1 }\right) + n     \\
             & = cn
            \left( \lg n - 1 - \lg \frac{ n }{ n - 1 } + \frac{ 1 }{ c }\right)
            - c \left( \lg \left( n - 1\right) - 1\right)             \\
             & \geqslant cn \left( \lg n - 2 + \frac{ 1 }{ c } -
            \frac{ \lg (n - 1) - 1 }{ n } \right)                     \\
             & \geqslant cn \left( \lg n - 3 + \frac{ 1 }{ c }\right) \\
             & \geqslant cn\lg n
        \end{aligned}
    \end{equation}
    因此 $\Omega (n\lg n)$ 也是这个递归式的解. 得到解为 $\Theta(n \lg n)$.

\end{solution}

%%%% Problem 4.3-5 %%%%
\problemnumber{5}
\begin{problem}
证明: 归并排序的 ``严格'' 递归式~\hyperref[eq:4.3-5.1]{(4.3)} 的解为 $\Theta(n
    \lg n)$.
\begin{equation}
    \label{eq:4.3-5.1}
    T(n) =
    \begin{cases}
        \Theta(1) & \text{若}\ n = 1 \\
        T \left( \left\lceil \dfrac{n}{2} \right\rceil \right) + T \left(
        \left\lfloor \dfrac{n}{2} \right\rfloor \right) + \Theta(n)
                  & \text{若}\ n > 1
    \end{cases}
\end{equation}
\end{problem}
\begin{solution}
    不妨令递归式中的 $\Theta(n)$ 满足 $d_{1}n \leqslant \Theta(n)
        \leqslant d_{2}n$.

    先证明 $O(n\lg n)$ 是这个递归式的解. 恰当选择常数
    $c \geqslant \frac{ 3 }{ 2 }d_{2}$, 有$T(n) \leqslant cn\lg n$. 假定此上界对
    所有正数 $m < n$ 都成立, 将此代入递归式, 若 $n$ 为偶数, 有
    \begin{equation}
        \begin{aligned}
            T(n)
             & = 2T \left( \frac{ n }{ 2 }\right) + \Theta(n) \\
             & \leqslant cn\lg \frac{ n }{ 2 } + d_{2}n       \\
             & = cn\lg n + \left( d_{2} - c\right)n           \\
             & \leqslant cn\lg n
        \end{aligned}
    \end{equation}
    若 $n$ 为奇数 (显然 $n \geqslant 3$), 有
    \begin{equation}
        \begin{aligned}
            T(n)
             & = T \left( \frac{ n - 1 }{ 2 }\right) +
            T \left( \frac{ n + 1 }{ 2 }\right) + \Theta(n)                \\
             & \leqslant c \frac{ n - 1 }{ 2 } \lg \frac{ n - 1 }{ 2 } +
            c \frac{ n + 1 }{ 2 } \lg \frac{ n + 1 }{ 2 } + d_{2} n        \\
             & = \frac{ cn }{ 2 } \lg \frac{ n^{2} - 1 }{ 4 } +
            \frac{ c }{ 2 }\lg \frac{ n + 1 }{ n - 1 } + d_{2}n            \\
             & \leqslant cn \lg n + c \frac{ 2 }{ n - 1 } - cn + d_{2}n    \\
             & = cn \lg n + d_{2} n - c \frac{ (n - 2)(n + 1) }{ (n - 1) } \\
             & \leqslant cn \lg n - d_{2}n \left( \frac{ 3 }{ 2 } \cdot
            \frac{ (n - 2)(n + 1) }{ (n - 1)n } - 1\right)                 \\
             & \leqslant cn \lg n
        \end{aligned}
    \end{equation}
    其中当 $n \geqslant 3$ 时,
    $\dfrac{ (n - 2)(n + 1) }{ (n - 1)n } \geqslant \dfrac{ 3 }{ 2 }$.

    再证明 $\Omega(n\lg n)$ 是这个递归式的解. 恰当选择常数
    $0 < c \leqslant \frac{ 16 }{ 17 }d_{1}$, 有 $T(n) \geqslant cn\lg n$. 假定
    此下界对所有正数 $m < n$ 都成立, 将此代入递归式, 若 $n$ 为偶数, 有
    \begin{equation}
        \begin{aligned}
            T(n)
             & = 2T \left( \frac{ n }{ 2 }\right) + \Theta(n) \\
             & \geqslant cn\lg \frac{ n }{ 2 } + d_{1}n       \\
             & = cn\lg n + \left( d_{1} - c\right)n           \\
             & \geqslant cn\lg n
        \end{aligned}
    \end{equation}
    若 $n$ 为奇数 (显然 $n \geqslant 3$), 有
    \begin{equation}
        \begin{aligned}
            T(n)
             & = T \left( \frac{ n - 1 }{ 2 }\right) +
            T \left( \frac{ n + 1 }{ 2 }\right) + \Theta(n)              \\
             & \geqslant c \frac{ n - 1 }{ 2 } \lg \frac{ n - 1 }{ 2 } +
            c \frac{ n + 1 }{ 2 } \lg \frac{ n + 1 }{ 2 } + d_{1} n      \\
             & = \frac{ cn }{ 2 } \lg \frac{ n^{2} - 1 }{ 4 } +
            \frac{ c }{ 2 }\lg \frac{ n + 1 }{ n - 1 } + d_{1}n          \\
             & \geqslant \frac{ cn }{ 2 }
            \left( \lg n^{2} - \lg \frac{ n^{2} }{ n^{2} - 1 } \right) -
            cn + d_{1}n                                                  \\
             & \geqslant cn\lg n -
            cn \left( 1 + \frac{ 1 }{ 2 \left( n^{2} - 1\right) }\right) +
            d_{1}n                                                       \\
             & \geqslant cn\lg n
        \end{aligned}
    \end{equation}
    其中当 $n \geqslant 3$ 时, $\dfrac{ 1 }{ 2 \left( n^{2} - 1\right) }
        \leqslant \dfrac{ 1 }{ 16 }$, 所以有最后一步的大于等于.

    综上所述, $O(n\lg n)$ 和 $\Omega(n\lg n)$ 都是递归式的解, 因此递归式的解为
    $\Theta(n\lg n)$
\end{solution}

%%%% Problem 4.3-7 %%%%
\problemnumber{7}
\begin{problem}
使用 4.5 节中的主方法, 可以证明 $T(n) = 4T \left( \frac{ n }{ 3 }\right) + n$ 的
的解为 $T(n) = \Theta \left( n^{\log_{3} 4}\right)$. 说明基于假设 $T(n)
    \leqslant c n^{\log_{3} 4}$ 的代入法不能证明这一结论. 然后说明如何通过减去一
个低阶项完成代入法证明.
\end{problem}
\begin{solution}
    基于假设 $T(n) \leqslant c n^{\log_{3} 4}$, 假定此上界对所有 $m < n$ 都成立,
    代入递归式, 得到
    \begin{equation}
        \begin{aligned}
            T(n)
             & \leqslant 4c {\left( \frac{ n }{ 3 }\right)}^{\log_{3} 4} + n \\
             & = cn^{\log_{3} 4} \cdot \frac{ 4 }{ 3^{\log_{3} 4} } + n      \\
             & = cn^{\log_{3} 4} + n
        \end{aligned}
    \end{equation}
    其中除了第一部是基于假设的直接放缩, 其他两步都是严格相等, 但是最后得到的结果
    为 $cn^{\log_{3} 4} + n > cn^{\log_{3} 4}$, 因此基于该假设无法证明该结论.

    不妨假设 $T(n) \leqslant cn^{\log_{3} 4} - 3n$, 只要 $c$ 取的足够大就能满足
    基本情况. 假定此上界对所有正数 $m < n$ 都成立, 代入递归式, 得到
    \begin{equation}
        \begin{aligned}
            T(n)
             & \leqslant 4
            \left( c {\left( \frac{ n }{ 3 }\right)}^{\log_{3} 4} - n\right)
            + n                           \\
             & = cn^{\log_{3} 4} - 4n + n \\
             & = cn^{\log_{3} 4} - 3n
        \end{aligned}
    \end{equation}
    因为 $cn^{\log_{3} 4} - 3n = O \left( n^{\log_{3} 4}\right)$, 所以
    $O \left( n^{\log_{3} 4}\right)$ 为递归式的解.

    再假设 $T(n) \geqslant cn^{\log_{3} 4}$. 假定此下界对所有正数 $m < n$ 都成
    立, 代入递归式, 得到
    \begin{equation}
        \begin{aligned}
            T(n)
             & \geqslant 4c {\left( \frac{ n }{ 3 }\right)}^{\log_{3} 4} + n \\
             & = cn^{\log_{3} 4} \cdot \frac{ 4 }{ 3^{\log_{3} 4} } + n      \\
             & = cn^{\log_{3} 4} + n                                         \\
             & \geqslant cn^{\log_{3} 4}
        \end{aligned}
    \end{equation}
    因此 $\Omega \left( n^{\log_{3} 4}\right)$ 是递归式的解.

    综上所述, $\Theta \left( n^{\log_{3} 4}\right)$ 是递归式的解.
\end{solution}

%%%% Problem 4.4-3 %%%%
\problempart{4}
\problemnumber{3}
\begin{problem}
对递归式 $T(n) = 4 T \left( \frac{ n }{ 2 } + 2\right) + n$, 利用递归树确定一个
好的渐进上界, 用代入法进行验证.
\end{problem}
\begin{solution}
    递归树的深度约为 $\lg n$, 深度为 $i$ 的层共有 $4^{i}$ 个结点, 每个结点代价约
    为 $\frac{ n }{ 2^{i} }$. 求和得到总代价约为 $2 n^{2}$, 因此假设渐进上界为
    $O \left( n^{2}\right)$. 使用代入法, 恰当选择常数 $c \geqslant 1$,
    有 $T(n) \leqslant c \left( n^{2} - 9n\right)$. 假定此上界对所有正数
    $m < n$ 都成立, 代入递归式, 得到
    \begin{equation}
        \begin{aligned}
            T(n)
             & \leqslant 4 c \left( {\left( \frac{ n }{ 2 } + 2\right)}^{2}
            - 9 \left( \frac{ n }{ 2 } + 2\right)\right) + n                \\
             & = c \left( n^{2} - 10n - 56\right) + n                       \\
             & = c \left( n^{2} - 9n\right) - (c - 1) n - 56c               \\
             & \leqslant c \left( n^{2} - 9n\right)
        \end{aligned}
    \end{equation}
    因为 $n^{2} - 9n = O \left( n^{2}\right)$, 所以 $O \left( n^{2}\right)$ 是
    渐进上界.
\end{solution}

%%%% Problem 4.4-5 %%%%
\problemnumber{5}
\begin{problem}
对递归式 $T(n) = T(n - 1) + T \left( \frac{ n }{ 2 }\right) + n$, 利用递归树确定
一个好的渐进上界, 用代入法进行验证.
\end{problem}
\begin{solution}
    递归树有一部分类似于 $T(n) = 2T(n - 1) + n$ 的递归树, 但是有另一部分类似于
    $T(n) = 2 \left( \frac{ n }{ 2 }\right) + n$ 的递归树. 前者的总代价是指数级
    别的 $O \left( 2^{n}\right)$, 所以不难得到解的一个渐进上界为
    $O \left( 2^{n}\right)$.

    使用代入法, 恰当选择常数 $c \geqslant 1$, 有 $T(n) \leqslant c 2^{n} - 3$.
    假定此上界对所有正数 $m < n$ 都成立, 代入递归式, 得到
    \begin{equation}
        \begin{aligned}
            T(n)
             & \leqslant c 2^{n - 1} - 3 + c 2^{ \frac{ n }{ 2 } } - 3 + n \\
             & = c \left( 2^{n - 1} + 2^{ \frac{ n }{ 2 } }\right) - 3 +
            (n - 3)                                                        \\
             &
            \begin{cases}
                \leqslant \displaystyle
                c \left( 2^{n - 1} + 2^{ \frac{ n }{ 2 } }\right) - 3
                    & 0 < n \leqslant 3 \\
                \leqslant \displaystyle
                c 2^{n - 1} + \left( c 2^{ \frac{ n }{ 2 } } + n\right) - 3
                \leqslant c \left( 2^{n - 1} + 2^{ \frac{ n }{ 2 } + 1}\right)
                - 3 & n \geqslant 4
            \end{cases}
            \\
             & \leqslant c 2^{n} - 3
        \end{aligned}
    \end{equation}
    因为 $c 2^{n} - 3 = O \left( 2^{n}\right)$, 所以 $O \left( 2^{n}\right)$ 是
    $T(n)$ 的递归上界.
\end{solution}

%%%% Problem 4.4-7 %%%%
\problemnumber{7}
\begin{problem}
对递归式 $T(n) = 4T \left( \left\lfloor \frac{ n }{ 2 }\right\rfloor\right)+ cn$
($c$ 为常数), 画出递归树, 并给出其解的一个渐进确界. 用代入法进行验证.
\end{problem}
\begin{solution}
    递归树如下图~\ref{fig:recursion-tree-4.4-7}所示.
    \begin{figure}[ht]
        \centering
        \begin{tikzpicture}[
                level 1/.style={sibling distance=3.6cm},
                level 2/.style={sibling distance=9mm},
                level 3/.style={sibling distance=2mm},
                ending/.style={edge from parent/.style={dashed,draw}},
                addup/.style={->,-latex,very thick,densely dotted,draw}
            ]
            \node (r1){$cn$}
            child {
                    node{$c \left\lfloor \frac{ n }{ 2 } \right\rfloor$}
                    child{
                            node{$c \left\lfloor \frac{ n }{ 4 } \right\rfloor$}
                            child[ending] {node (l) {}
                                    child [grow=left, level distance=5mm] {node (b) {} edge from parent[draw = none]
                                            child [grow=up, level distance=1.5cm] {node () {} edge from parent[draw = none]
                                                    child [grow=up] {node () {} edge from parent[draw = none]
                                                            child [grow=up] {node (t) {} edge from parent[draw = none]
                                                                }
                                                        }
                                                }
                                        }
                                }
                            child[ending] {node {}}
                            child[ending] {node {}}
                            child[ending] {node {}}
                        }
                    child{
                            node{$c \left\lfloor \frac{ n }{ 4 } \right\rfloor$}
                            child[ending] {node {}}
                            child[ending] {node {}}
                            child[ending] {node {}}
                            child[ending] {node {}}
                        }
                    child{
                            node{$c \left\lfloor \frac{ n }{ 4 } \right\rfloor$}
                            child[ending] {node {}}
                            child[ending] {node {}}
                            child[ending] {node {}}
                            child[ending] {node {}}
                        }
                    child{
                            node{$c \left\lfloor \frac{ n }{ 4 } \right\rfloor$}
                            child[ending] {node {}}
                            child[ending] {node {}}
                            child[ending] {node {}}
                            child[ending] {node {}}
                        }
                }
            child {
                    node{$c \left\lfloor \frac{ n }{ 2 } \right\rfloor$}
                    child{
                            node{$c \left\lfloor \frac{ n }{ 4 } \right\rfloor$}
                            child[ending] {node {}}
                            child[ending] {node {}}
                            child[ending] {node {}}
                            child[ending] {node {}}
                        }
                    child{
                            node{$c \left\lfloor \frac{ n }{ 4 } \right\rfloor$}
                            child[ending] {node {}}
                            child[ending] {node {}}
                            child[ending] {node {}}
                            child[ending] {node {}}
                        }
                    child{
                            node{$c \left\lfloor \frac{ n }{ 4 } \right\rfloor$}
                            child[ending] {node {}}
                            child[ending] {node {}}
                            child[ending] {node {}}
                            child[ending] {node {}}
                        }
                    child{
                            node{$c \left\lfloor \frac{ n }{ 4 } \right\rfloor$}
                            child[ending] {node {}}
                            child[ending] {node {}}
                            child[ending] {node {}}
                            child[ending] {node {}}
                        }
                }
            child {
                    node{$c \left\lfloor \frac{ n }{ 2 } \right\rfloor$}
                    child{
                            node{$c \left\lfloor \frac{ n }{ 4 } \right\rfloor$}
                            child[ending] {node {}}
                            child[ending] {node {}}
                            child[ending] {node {}}
                            child[ending] {node {}}
                        }
                    child{
                            node{$c \left\lfloor \frac{ n }{ 4 } \right\rfloor$}
                            child[ending] {node {}}
                            child[ending] {node {}}
                            child[ending] {node {}}
                            child[ending] {node {}}
                        }
                    child{
                            node{$c \left\lfloor \frac{ n }{ 4 } \right\rfloor$}
                            child[ending] {node {}}
                            child[ending] {node {}}
                            child[ending] {node {}}
                            child[ending] {node {}}
                        }
                    child{
                            node{$c \left\lfloor \frac{ n }{ 4 } \right\rfloor$}
                            child[ending] {node {}}
                            child[ending] {node {}}
                            child[ending] {node {}}
                            child[ending] {node {}}
                        }
                }
            child {
                    node (r2) {$c \left\lfloor \frac{ n }{ 2 } \right\rfloor$}
                    child{
                            node{$c \left\lfloor \frac{ n }{ 4 } \right\rfloor$}
                            child[ending] {node {}}
                            child[ending] {node {}}
                            child[ending] {node {}}
                            child[ending] {node {}}
                        }
                    child{
                            node{$c \left\lfloor \frac{ n }{ 4 } \right\rfloor$}
                            child[ending] {node {}}
                            child[ending] {node {}}
                            child[ending] {node {}}
                            child[ending] {node {}}
                        }
                    child{
                            node{$c \left\lfloor \frac{ n }{ 4 } \right\rfloor$}
                            child[ending] {node {}}
                            child[ending] {node {}}
                            child[ending] {node {}}
                            child[ending] {node {}}
                        }
                    child{
                            node (r3) {$c \left\lfloor \frac{ n }{ 4 } \right\rfloor$}
                            child[ending] {node {}}
                            child[ending] {node {}}
                            child[ending] {node {}}
                            child[ending] {node (r) {}
                                    child [grow=right, level distance=1cm] {node (rb) {$\vdots$} edge from parent[draw = none]
                                            child [grow=up, level distance=1.5cm] {node (s3) {$4cn$} edge from parent[draw = none]
                                                    child [grow=up] {node (s2) {$2cn$} edge from parent[draw = none]
                                                            child [grow=up] {node (s1) {$cn$} edge from parent[draw = none]
                                                                }
                                                        }
                                                }
                                        }
                                }
                        }
                };
            \node[below of=rb, anchor=east] {总计: $O \left( n^{2}\right)$};
            \path (l) -- (r) node [midway] {$\vdots$};
            \path (r1) edge[addup] (s1);
            \path (r2) edge[addup] (s2);
            \path (r3) edge[addup] (s3);
            \draw[<->, >=latex] (b) edge node[left] {$\lg n$} (t);
        \end{tikzpicture}
        \caption{题中递归式对应的递归树}
        \label{fig:recursion-tree-4.4-7}
    \end{figure}
    得到渐进确界为 $\Theta \left( n^{2}\right)$, 下面用代入法证明.

    先证明 $T(n) = O \left( n^{2}\right)$. 恰当选择常数 $d > c$, 有
    $T(n) \leqslant dn^{2} - cn$. 假定此上界对所有正数 $m < n$ 都成立, 将其代入
    递归式得到
    \begin{equation}
        \begin{aligned}
            T(n)
             & \leqslant 4 d{\left\lfloor \frac{ n }{ 2 }\right\rfloor}^{2} -
            4c \left\lfloor \frac{ n }{ 2 }\right\rfloor + cn                 \\
             & \leqslant 4 d{\left( \frac{ n }{ 2 }\right)}^{2} - cn          \\
             & = dn^{2} - cn
        \end{aligned}
    \end{equation}

    再证明 $T(n) = \Omega \left( n^{2}\right)$. 恰当选择常数
    $\frac{ c }{ 5 } \geqslant d > 0$, 有
    $T(n) \geqslant d \left( n^{2} + 4n\right)$. 假定此下界对所有正数 $m < n$
    都成立, 将其代入递归式得到
    \begin{equation}
        \begin{aligned}
            T(n)
             & \geqslant 4d \left(
            {\left\lfloor \frac{ n }{ 2 }\right\rfloor}^{2} +
            4 \left\lfloor \frac{ n }{ 2 }\right\rfloor \right) + cn  \\
             & \geqslant 4d {\left( \frac{ n - 1 }{ 2 }\right)}^{2} +
            8d(n - 1) + cn                                            \\
             & = d \left( n^{2} +4n\right) + 2dn + cn - 7d            \\
             & \geqslant d \left( n^{2} +4n\right) + cn - 5d          \\
             & \geqslant d \left( n^{2} + 4n\right)
        \end{aligned}
    \end{equation}
\end{solution}

%%%% Problem 4.5-2 %%%%
\problempart{5}
\problemnumber{2}
\begin{problem}
Caesar 教授项设计一个渐进快于 Strassen 算法的矩阵相乘算法. 他的算法使用分治方法,
将每个矩阵分解为 $\frac{ n }{ 4 } \times \frac{ n }{ 4 }$ 的子矩阵, 分解和合并步
骤共花费 $\Theta \left( n^{2}\right)$ 时间. 他需要确定, 他的算法需要创建多少个子
问题, 才能击败 Strassen 算法. 如果他的算法创建 $a$ 个子问题, 则描述运行时间
$T(n)$ 的递归式为 $T(n) = aT \left( \frac{ n }{ 4 }\right) + \Theta(n^{2})$.
Caesar 教授的算法如果要渐进快于 Strassen 算法, $a$ 的最大整数值应是多少?
\end{problem}
\begin{solution}
    当 $a$ 取最大值时, 递归式显然适用于主方法的情况一, 递归式的解为
    \begin{equation}
        T(n) = \Theta \left( n^{ \frac{ 1 }{ 2 }\lg n} \right) < \Theta
        \left( n^{\lg 7}\right)
    \end{equation}
    解得 $a < 49$, 最大值整数值为 48.

    当 $a$ 为 48 时, 对于常数 $\varepsilon = \frac{ 1 }{ 2 }\lg 48 - 2 > 0$, 有
    $f(n) = \Theta\left( n^{2}\right) =
        \Theta \left( n^{\log_{4} 48 - \varepsilon}\right)$, 满足主方法情况一的
    条件.
\end{solution}

%%%% Problem 4.5-4 %%%%
\problemnumber{4}
\begin{problem}
主方法能应用于递归式 $T(n) = 4T \left( \frac{ n }{ 2 }\right) + n^{2}\lg n$ 吗?
请说明为什么可以或者为什么不可以. 给出这个递归式的一个渐进上界.
\end{problem}
\begin{solution}
    可以应用 (如果是顾老师 PPT 上的主方法). 递归式满足 $f(n) = \Theta
        \left( n^{\log_{2} 4}\lg^{1} n \right)$, 由主方法情况二可得
    $T(n) = \Theta \left( n^{2}\lg^{2} n\right)$.
\end{solution}

% \begin{algo}
%     \caption{INSERTION-SORT($A$)}
%     \label{algo:insertion-sort}
%     \For{$j = 2$ \textbf{to} $A.length$}{
%         $key = A[j]$\;
%         \tcp{Insert $A[j]$ into the sorted sequence $A[1..j-1]$.}
%         $i = j - 1$\;
%         \While{$i > 0$ and $A[i] > key$}{
%             $A[i+1] = A[i]$\;
%             $i = i - 1$\;
%         }
%         $A[i+1] = key$\;
%     }
% \end{algo}

\end{document}
