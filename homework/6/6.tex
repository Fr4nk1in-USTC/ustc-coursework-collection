\documentclass[boxes]{homework}

% This is a slightly-more-than-minimal document that uses the homework class.
% See the README at http://git.io/vZWL0 for complete documentation.

\name{傅申 PB20000051}        % Replace (Your Name) with your name.
\term{2022 秋}     % Replace (Current Term) with the current term.
\course{编译原理和技术}    % Replace (Course Name) with the course name.
\hwnum{6}          % Replace (Number) with the number of the homework.
\hwname{作业}
\problemname{习题}
\solutionname{解:}
\problemchap{7}

% Load any other packages you need here.
\usepackage[
    a4paper,
    top = 2.54cm,
    bottom = 2.54cm,
    left = 1.91cm,
    right = 1.91cm,
    includeheadfoot
]{geometry}
\fancyfootoffset{0pt} % make fancyhdr work properly
\usepackage{ctex}

\begin{document}
%%%% Problem 7.1 (d) %%%%
\begin{problem}
把算术表达式 $-(a + b) * (c + d) + (a + b + c)$ 翻译成:
\begin{parts}
    \setcounter{enumi}{3}
    \part\label{prob:7.1.d} 三地址代码
\end{parts}
\end{problem}
\begin{solution}
    \ref{prob:7.1.d} 如下
    \begin{verbatim}
        t1 = a + b
        t2 = - t1
        t3 = c + d
        t4 = t2 * t3
        t5 = t1 + c
        t6 = t4 + t5   
    \end{verbatim}
\end{solution}

%%%% Problem 7.2 (c) %%%%
\begin{problem}
把 C 程序
\begin{verbatim}
    main(){
        int i;
        int a[10];
        while(i<=10)
            a[i] = 0;
    }
\end{verbatim}
的可执行语句翻译成:
\begin{parts}
    \setcounter{enumi}{2}
    \part\label{prob:7.2.c} 三地址代码
\end{parts}
\end{problem}
\begin{solution}
    \ref{prob:7.2.c} 如下, 假设每个 \texttt{int} 占 4 个存储单元.
    \begin{verbatim}
        L1: t1 = i <= 10
            if t1 goto L2, if False t1 goto L3

        L2: t2 = i * 4
            a[t2] = 0
            goto L1

        L3: return
    \end{verbatim}
\end{solution}

%%%% Problem 7.5 %%%%
\problemnumber{5}
\begin{problem}
修改图~\ref{eq:7.5} 中计算声明名字的类型和相对地址的翻译方案, \textit{offset} 不
是全局变量, 而是文法符号的继承属性.
\begin{equation}
    \begin{aligned}
        P \to &                                                    &  & \{\mathit{offset} = 0;\}                                                                            \\
              & D; S                                                                                                                                                        \\
        D \to & D; D                                                                                                                                                        \\
        D \to & \mathbf{id}: T                                     &  & \{enter(\mathbf{id}.lexme, T.type, \mathit{offset}); \mathit{offset} = \mathit{offset} + T.width;\} \\
        T \to & \mathbf{integer}                                   &  & \{T.type = integer; T.width = 4;\}                                                                  \\
        T \to & \mathbf{real}                                      &  & \{T.type = real; T.width = 8;\}                                                                     \\
        T \to & \mathbf{array}\ [\mathbf{num}]\ \mathbf{of}\ T_{1} &  & \{T.type = array(\mathbf{num}.val, T_{1}.type);                                                     \\
              &                                                    &  & \qquad T.width = \mathbf{num}.val \times T_{1}.width;\}                                             \\
        T \to & \uparrow T_{1}                                     &  & \{T.type = pointer(T_{1}.type); T.width = 4;\}
    \end{aligned}
    \tag{7.5}
    \label{eq:7.5}
\end{equation}
\end{problem}
\begin{solution}
    如下, 其中 \textit{end} 是综合属性:
    \begin{equation}
        \begin{aligned}
            P \to &                                                    &  & \{D.\mathit{offset} = 0;\}                                                                    \\
                  & D; S                                                                                                                                                  \\
            D \to &                                                    &  & \{D_{1}.\mathit{offset} = D.\mathit{offset};\}                                                \\
                  & D_{1};                                             &  & \{D_{2}.\mathit{offset} = D_{1}.end;\}                                                        \\
                  & D_{2}                                                                                                                                                 \\
            D \to & \mathbf{id}: T                                     &  & \{enter(\mathbf{id}.lexme, T.type, D.\mathit{offset}); D.end = D.\mathit{offset} + T.width;\} \\
            T \to & \mathbf{integer}                                   &  & \{T.type = integer; T.width = 4;\}                                                            \\
            T \to & \mathbf{real}                                      &  & \{T.type = real; T.width = 8;\}                                                               \\
            T \to & \mathbf{array}\ [\mathbf{num}]\ \mathbf{of}\ T_{1} &  & \{T.type = array(\mathbf{num}.val, T_{1}.type);                                               \\
                  &                                                    &  & \qquad T.width = \mathbf{num}.val \times T_{1}.width;\}                                       \\
            T \to & \uparrow T_{1}                                     &  & \{T.type = pointer(T_{1}.type); T.width = 4;\}
        \end{aligned}
    \end{equation}
\end{solution}

\end{document}
