\documentclass[boxes]{homework}

% This is a slightly-more-than-minimal document that uses the homework class.
% See the README at http://git.io/vZWL0 for complete documentation.

\name{傅申 PB20000051}        % Replace (Your Name) with your name.
\term{2022 秋}     % Replace (Current Term) with the current term.
\course{编译原理和技术}    % Replace (Course Name) with the course name.
\hwnum{7}          % Replace (Number) with the number of the homework.
\hwname{作业}
\problemname{习题}
\solutionname{解:}
\problemchap{9}

% Load any other packages you need here.
\usepackage[
    a4paper,
    top = 2.54cm,
    bottom = 2.54cm,
    left = 1.91cm,
    right = 1.91cm,
    includeheadfoot
]{geometry}
\fancyfootoffset{0pt} % make fancyhdr work properly
\usepackage{ctex}

\begin{document}

\problemnumber{3}
\begin{problem}
对图 9.32 的流图, 计算:
\begin{parts}
    \part\label{prob:9.3.a}
    为到达-定值分析, 计算每个块的 $gen$, $kill$, IN 和 OUT 集合.
    \part\label{prob:9.3.b}
    为可用表达式分析, 计算每个块的 $e\_gen$, $e\_kill$, IN 和 OUT 集合.
    \part\label{prob:9.3.c}
    为活跃变量分析, 计算每个块的 $def$, $use$, IN 和 OUT 集合.
\end{parts}
\end{problem}
\begin{solution}
    \ref{prob:9.3.a} 先求 $gen$ 和 $kill$ 集合, 如下
    \begin{center}
        \begin{tabular}{|c|c|c|}
            \hline
            基本块  & $gen$        & $kill$          \\ \hline
            $B_{1}$ & $\{1, 2\}$   & $\{8, 10, 11\}$ \\ \hline
            $B_{2}$ & $\{3, 4\}$   & $\{5, 6\}$      \\ \hline
            $B_{3}$ & $\{5\}$      & $\{4, 6\}$      \\ \hline
            $B_{4}$ & $\{6, 7\}$   & $\{4, 5, 9\}$   \\ \hline
            $B_{5}$ & $\{8, 9\}$   & $\{2, 7, 11\}$  \\ \hline
            $B_{6}$ & $\{10, 11\}$ & $\{1, 2, 8\}$   \\ \hline
        \end{tabular}
    \end{center}
    然后迭代求 IN 和 OUT 集合, 如下
    \begin{center}
        \begin{tabular}{|c|c|c|c|c|c|}
            \hline
            基本块  & $\mathrm{OUT}^{0}$ & $\mathrm{IN}^{1}$           & $\mathrm{OUT}^{1}$          & $\mathrm{IN}^{2}$            & $\mathrm{OUT}^{2}$        \\ \hline
            $B_{1}$ & $\varnothing$      & $\varnothing$               & $\{1, 2\}$                  & $\varnothing$                & $\{1, 2\}$                \\ \hline
            $B_{2}$ & $\varnothing$      & $\{1, 2\}$                  & $\{1, 2, 3, 4\}$            & $\{1, 2, 3, 4, 5, 8, 9\}$    & $\{1, 2, 3, 4, 8, 9\}$    \\ \hline
            $B_{3}$ & $\varnothing$      & $\{1, 2, 3, 4\}$            & $\{1, 2, 3, 5\}$            & $\{1, 2, 3, 4, 6, 7, 8, 9\}$ & $\{1, 2, 3, 5, 7, 8, 9\}$ \\ \hline
            $B_{4}$ & $\varnothing$      & $\{1, 2, 3, 5\}$            & $\{1, 2, 3, 6, 7\}$         & $\{1, 2, 3, 5, 7, 8, 9\}$    & $\{1, 2, 3, 6, 7, 8\}$    \\ \hline
            $B_{5}$ & $\varnothing$      & $\{1, 2, 3, 4, 5\}$         & $\{1, 3, 4, 5, 8, 9\}$      & $\{1, 2, 3, 4, 5, 7, 8, 9\}$ & $\{1, 3, 4, 5, 8, 9\}$    \\ \hline
            $B_{6}$ & $\varnothing$      & $\{1, 3, 4, 5, 8, 9\}$      & $\{3, 4, 5, 8, 9, 10, 11\}$ & $\{1, 3, 4, 5, 8, 9\}$       & $\{3, 4, 5, 9, 10, 11\}$  \\ \hline
            EXIT    & $\varnothing$      & $\{3, 4, 5, 8, 9, 10, 11\}$ & $\{3, 4, 5, 8, 9, 10, 11\}$ & $\{3, 4, 5, 9, 10, 11\}$     & $\{3, 4, 5, 9, 10, 11\}$  \\ \hline
        \end{tabular}
    \end{center}
    最后两列即为最终得到的 IN 和 OUT 集合.

    \ref{prob:9.3.b} 全体表达式为 $U = $ \{1, 2, a+b, c-a, b+d, e+1, b*d, a-d\},
    先求 $e\_gen$ 和 $e\_kill$ 集合, 如下
    \begin{center}
        \begin{tabular}{|c|c|c|}
            \hline
            基本块  & $e\_gen$      & $e\_kill$                   \\ \hline
            $B_{1}$ & \{1, 2\}      & \{a+b, c-a, b+d, b*d, a-d\} \\ \hline
            $B_{2}$ & \{a+b, c-a\}  & \{c-a, b+d, b*d, a-d\}      \\ \hline
            $B_{3}$ & $\varnothing$ & \{b+d, b*d, a-d\}           \\ \hline
            $B_{4}$ & \{a+b\}       & \{b+d, e+1, b*d, a-d\}      \\ \hline
            $B_{5}$ & \{c-a\}       & \{a+b, b+d, e+1, b*d\}      \\ \hline
            $B_{6}$ & \{a-d\}       & \{a+b, c-a, b+d, b*d\}      \\ \hline
        \end{tabular}
    \end{center}
    然后迭代求 IN 和 OUT 集合, 如下
    \begin{center}
        \begin{tabular}{|c|c|c|c|}
            \hline
            基本块  & $\mathrm{OUT}^{0}$ & $\mathrm{IN}^{1}$  & $\mathrm{OUT}^{1}$ \\ \hline
            $B_{1}$ & $U$                & $\varnothing$      & \{1, 2\}           \\ \hline
            $B_{2}$ & $U$                & \{1, 2\}           & \{1, 2, a+b, c-a\} \\ \hline
            $B_{3}$ & $U$                & \{1, 2, a+b, c-a\} & \{1, 2, a+b, c-a\} \\ \hline
            $B_{4}$ & $U$                & \{1, 2, a+b, c-a\} & \{1, 2, a+b, c-a\} \\ \hline
            $B_{5}$ & $U$                & \{1, 2, a+b, c-a\} & \{1, 2, c-a\}      \\ \hline
            $B_{6}$ & $U$                & \{1, 2, c-a\}      & \{1, 2, a-d\}      \\ \hline
            EXIT    & $U$                & \{1, 2, a-d\}      & \{1, 2, a-d\}      \\ \hline
        \end{tabular}
    \end{center}

    \ref{prob:9.3.c} 全体变量集合为 \{a, b, c, d, e\}, 先求 $use$ 和 $def$ 集合,
    如下
    \begin{center}
        \begin{tabular}{|c|c|c|}
            \hline
            基本块  & $use$         & $def$         \\ \hline
            $B_{1}$ & $\varnothing$ & \{a, b\}      \\ \hline
            $B_{2}$ & \{a, b\}      & \{c, d\}      \\ \hline
            $B_{3}$ & \{b, d\}      & $\varnothing$ \\ \hline
            $B_{4}$ & \{a, b, e\}   & \{d\}         \\ \hline
            $B_{5}$ & \{a, b, c\}   & \{e\}         \\ \hline
            $B_{6}$ & \{b, d\}      & \{a\}         \\ \hline
        \end{tabular}
    \end{center}
    然后迭代求 IN 和 OUT 集合, 计算次序为 $B_{6} \to B_{5} \to B_{3} \to B_{4}
        \to B_{2} \to B_{1}$, 如下
    \begin{center}
        \resizebox{.95\textwidth}{!}{
            \begin{tabular}{|c|c|c|c|c|c|c|c|}
                \hline
                基本块  & $\mathrm{IN}^{0}$ & $\mathrm{OUT}^{1}$ & $\mathrm{IN}^{1}$ & $\mathrm{OUT}^{2}$ & $\mathrm{IN}^{2}$ & $\mathrm{OUT}^{2}$ & $\mathrm{IN}^{2}$ \\ \hline
                ENTER   & $\varnothing$     & $\varnothing$      & $\varnothing$     & \{e\}              & \{e\}             & \{e\}              & \{e\}             \\ \hline
                $B_{1}$ & $\varnothing$     & \{a, b\}           & $\varnothing$     & \{a, b, e\}        & \{e\}             & \{a, b, e\}        & \{e\}             \\ \hline
                $B_{2}$ & $\varnothing$     & \{a, b, c, d\}     & \{a, b\}          & \{a, b, c, d, e\}  & \{a, b, e\}       & \{a, b, c, d, e\}  & \{a, b, e\}       \\ \hline
                $B_{3}$ & $\varnothing$     & \{a, b, c, d\}     & \{a, b, c, d\}    & \{a, b, c, d, e\}  & \{a, b, c, d, e\} & \{a, b, c, d, e\}  & \{a, b, c, d, e\} \\ \hline
                $B_{4}$ & $\varnothing$     & \{a, b, c, d\}     & \{a, b, c, e\}    & \{a, b, c, d, e\}  & \{a, b, c, e\}    & \{a, b, c, d, e\}  & \{a, b, c, e\}    \\ \hline
                $B_{5}$ & $\varnothing$     & \{b, d\}           & \{a, b, c, d\}    & \{a, b, d\}        & \{a, b, c, d\}    & \{a, b, d, e\}     & \{a, b, c, d\}    \\ \hline
                $B_{6}$ & $\varnothing$     & $\varnothing$      & \{b, d\}          & $\varnothing$      & \{b, d\}          & $\varnothing$      & \{b, d\}          \\ \hline
            \end{tabular}
        }
    \end{center}
    最后两列即为最终得到的 IN 和 OUT 集合.
\end{solution}

\problemnumber{22}
\begin{problem}
请利用代码优化的思想 (代码外提和强度削弱等), 改写下面 C 语言程序中的循环, 得到优
化后的 C 语言程序.
\begin{verbatim}
        main() {
            int i, j;
            int r[20][10];

            for (i = 0; i < 20; i++) {
                for (j = 0; j < 10; j++) {
                    r[i][j] = 10 * i * j;
                }
            }
        }
    \end{verbatim}
\end{problem}
\begin{solution}
    内循环中的 \texttt{i * 10} 是循环不变计算, 可以提至外循环中. 可以用加法代替
    乘法计算, 因为每次执行外循环 \texttt{i * 10} 都只增加了 10, 而每次执行内循环
    \texttt{10 * i * j} 都只增加了 \texttt{10 * i}. 同时可以用数组元素地址代替去
    下标运算, 因为每次执行内循环 \texttt{r[i][j]} 都是数组的下一个元素. 综合起
    来, 优化后的 C 语言程序为
    \begin{verbatim}
        main() {
            int i, j, m, n;
            int *ptr;
            int r[20][10];
            ptr = &r[0][0];
            m = 0; // m = 10 * i

            for (i = 0; i < 20; i++) {
                n = 0; // n = 10 * i * j
                for (j = 0; j < 10; j++) {
                    *ptr = n;
                    ptr++;
                    n = n + m;
                }
                m = m + 10;
            }
        }
    \end{verbatim}
\end{solution}


\end{document}
