\documentclass[boxes]{homework}

% This is a slightly-more-than-minimal document that uses the homework class.
% See the README at http://git.io/vZWL0 for complete documentation.

\name{傅申 PB20000051}        % Replace (Your Name) with your name.
\term{2022 秋}     % Replace (Current Term) with the current term.
\course{编译原理和技术 B}    % Replace (Course Name) with the course name.
\hwnum{1}          % Replace (Number) with the number of the homework.
\hwname{作业}
\problemname{}
\solutionname{解:}
\problemchap{2}

% Load any other packages you need here.
\usepackage[
    a4paper,
    top = 2.54cm,
    bottom = 2.54cm,
    left = 1.91cm,
    right = 1.91cm,
    includeheadfoot
]{geometry}
\fancyfootoffset{0pt} % make fancyhdr work properly
\usepackage{ctex}
\usepackage{multirow}

\allowdisplaybreaks[4]

\begin{document}
%%%% Problem 2.1 (a) %%%%
\begin{problem}
从下列每种语言的参考手册确定它们形成输入字母表的字符集 (不包括那些只可以出现在
字符串或注释中的字符).
\begin{parts}[a]
    \part\label{prob:2.1.a}
    C
\end{parts}
\end{problem}
\begin{solution}
    \ref{prob:2.1.a}
    从 \textit{C: A Reference Manual} 第五版第一部分第 2.1 节可得到形成 C 语言
    输入字母表的字符集包括
    \begin{itemize}
        \item 52 个大小写拉丁字母;
        \item 10 个阿拉伯数字;
        \item 空格;
        \item 水平制表符 HT, 垂直制表符 VT 和换页符 FF\@;
        \item 29 个特殊字符.
    \end{itemize}
    所以 C 语言输入字母表的字符集为 \{ \texttt{a}, \texttt{b}, \texttt{c},
    \texttt{d}, \texttt{e}, \texttt{f}, \texttt{g}, \texttt{h}, \texttt{i},
    \texttt{j}, \texttt{k}, \texttt{l}, \texttt{m}, \texttt{n}, \texttt{o},
    \texttt{p}, \texttt{q}, \texttt{r}, \texttt{s}, \texttt{t}, \texttt{u},
    \texttt{v}, \texttt{w}, \texttt{x}, \texttt{y}, \texttt{z}, \texttt{A},
    \texttt{B}, \texttt{C}, \texttt{D}, \texttt{E}, \texttt{F}, \texttt{G},
    \texttt{H}, \texttt{I}, \texttt{J}, \texttt{K}, \texttt{L}, \texttt{M},
    \texttt{N}, \texttt{O}, \texttt{P}, \texttt{Q}, \texttt{R}, \texttt{S},
    \texttt{T}, \texttt{U}, \texttt{V}, \texttt{W}, \texttt{X}, \texttt{Y},
    \texttt{Z}, \texttt{0}, \texttt{1}, \texttt{2}, \texttt{3}, \texttt{4},
    \texttt{5}, \texttt{6}, \texttt{7}, \texttt{8}, \texttt{9}, \texttt{!},
    \texttt{"}, \texttt{\#}, \texttt{\%}, \texttt{\&}, \texttt{'}, \texttt{(},
    \texttt{)}, \texttt{*}, \texttt{+}, \texttt{,}, \texttt{-}, \texttt{.},
    \texttt{/}, \texttt{:}, \texttt{;}, \texttt{<}, \texttt{=}, \texttt{>},
    \texttt{?}, \texttt{[}, \texttt{\textbackslash}, \texttt{]}, \texttt{\^{}},
    \texttt{\_}, \texttt{\{}, \texttt{|}, \texttt{\}}, \texttt{\textasciitilde},
    空格, 水平制表符 HT, 垂直制表符 VT, 换页符 FF \}
\end{solution}

%%%% Problem 2.3 (b) %%%%
\problemnumber{3}
\begin{problem}
叙述由下列正规式描述的语言.
\begin{parts}[a]
    \setcounter{enumi}{1}
    \part\label{prob:2.3.b}
    ${((\varepsilon | 0)1^{*})}^{*}$
\end{parts}
\end{problem}
\begin{solution}
    \ref{prob:2.3.b}
    因为 $\{\varepsilon, 0, 1\}\subset L((\varepsilon|0)1^{*})$, 而显然
    $L({((\varepsilon|0)1^{*})}^{*}) \subseteq L({(0 | 1)}^{*})$, 所以有
    $L({((\varepsilon|0)1^{*})}^{*}) = L({(0 | 1)}^{*})$. 即 ${((\varepsilon |
                0)1^{*})}^{*}$ 表示由 0 和 1 构成的所有串的集合, 包括空串.
\end{solution}

%%%% Problem 2.4 (b),(i) %%%%
\begin{problem}
为下列语言写出正规定义.
\begin{parts}[a]
    \setcounter{enumi}{1}
    \part\label{prob:2.4.b}
    按字典序排列的所有字母串.
    \setcounter{enumi}{8}
    \part\label{prob:2.4.i}
    字母表 $\{a, b\}$ 上, $a$ 不会相邻出现的所有串.
\end{parts}
\end{problem}
\begin{solution}
    \ref{prob:2.4.b}
    $answer\to A^{*}a^{*}B^{*}b^{*}\cdots Z^{*}z^{*}$

    \ref{prob:2.4.i}
    $answer\to a?{\left( b | ba\right)}^{*}$

\end{solution}

%%%% Problem 2.7 (c) %%%%
\problemnumber{7}
\begin{problem}
用算法 2.4 为下列正规式构造不确定有限自动机, 给出它们处理输入串 $ababbab$ 的
状态转换序列.
\begin{parts}[a]
    \setcounter{enumi}{2}
    \part\label{prob:2.7.c}
    ${((\varepsilon | a)b^{*})}^{*}$
\end{parts}
\end{problem}
\begin{solution}
    \ref{prob:2.7.c}
    不确定有限自动机如下图~\ref{fig:NFA-2.7.c}, 状态转换序列为 $0\to 1\to 2\to
        3\to 6\to 7\to 8\to 9\to 1\to 2\to 3\to 6\to 7\to 8\to 7\to 8\to 9\to 1
        \to 2\to 3\to 6\to 7\to 8\to 9\to 10$.
    \begin{figure}[ht]
        \centering
        \begin{tikzpicture}
            \node[state]                          (0)  {$0$};
            \node[left of=0, node distance=1.5cm] (n)  {};
            \node[state, right of=0]              (1)  {$1$};
            \node[state, above right of=1]        (2)  {$2$};
            \node[state, right of=2]              (3)  {$3$};
            \node[state, below right of=1]        (4)  {$4$};
            \node[state, right of=4]              (5)  {$5$};
            \node[state, above right of=5]        (6)  {$6$};
            \node[state, right of=6]              (7)  {$7$};
            \node[state, right of=7]              (8)  {$8$};
            \node[state, right of=8]              (9)  {$9$};
            \node[state, accepting, right of=9]   (10) {$10$};

            \draw
            (n) edge                    node[above] {开始}          (0)
            (0) edge                    node[above] {$\varepsilon$} (1)
            (0) edge[bend right=40]     node[above] {$\varepsilon$} (10)
            (1) edge                    node[above] {$\varepsilon$} (2)
            (1) edge                    node[above] {$\varepsilon$} (4)
            (2) edge                    node[above] {$a$}           (3)
            (3) edge                    node[above] {$\varepsilon$} (6)
            (4) edge                    node[above] {$\varepsilon$} (5)
            (5) edge                    node[above] {$\varepsilon$} (6)
            (6) edge                    node[above] {$\varepsilon$} (7)
            (6) edge[bend right=40]     node[above] {$\varepsilon$} (9)
            (7) edge[bend right=40]     node[above] {$b$}           (8)
            (8) edge[bend right=40]     node[above] {$\varepsilon$} (7)
            (8) edge                    node[above] {$\varepsilon$} (9)
            (9) edge[bend right=80]     node[above] {$\varepsilon$} (1)
            (9) edge                    node[above] {$\varepsilon$} (10)
            ;
        \end{tikzpicture}
        \caption{识别 ${((\varepsilon | a)b^{*})}^{*}$ 的 NFA}
        \label{fig:NFA-2.7.c}
    \end{figure}

\end{solution}

%%%% Problem 2.12 (a) %%%%
\problemnumber{12}
\begin{problem}
为下列正规式构造最简的 DFA\@.
\begin{parts}[a]
    \part\label{prob:2.12.a}
    ${(a|b)}^{*} a {(a|b)}$
\end{parts}
\end{problem}
\begin{solution}
    \ref{prob:2.12.a}
    首先构造 NFA 如下图~\ref{fig:NFA-2.12.a}.
    \begin{figure}[ht]
        \centering
        \begin{tikzpicture}
            \node[state]                                (0)  {$0$};
            \node[left of=0]                            (n)  {};
            \node[state, right of=0]                    (1)  {$1$};
            \node[state, above right of=1]              (2)  {$2$};
            \node[state, right of=2]                    (3)  {$3$};
            \node[state, below right of=1]              (4)  {$4$};
            \node[state, right of=4]                    (5)  {$5$};
            \node[state, above right of=5]              (6)  {$6$};
            \node[state, right of=6]                    (7)  {$7$};
            \node[state, right of=7]                    (8)  {$8$};
            \node[state, above right of=8]              (9)  {$9$};
            \node[state, right of=9]                    (10) {$10$};
            \node[state, below right of=8]              (11) {$11$};
            \node[state, right of=11]                   (12) {$12$};
            \node[state, accepting, above right of=12]  (13) {$13$};

            \draw
            (n)     edge                                node[above] {开始}          (0)
            (0)     edge                                node[above] {$\varepsilon$} (1)
            (0)     edge[bend right=80]                 node[above] {$\varepsilon$} (7)
            (1)     edge                                node[above] {$\varepsilon$} (2)
            (1)     edge                                node[above] {$\varepsilon$} (4)
            (2)     edge                                node[above] {$a$}           (3)
            (3)     edge                                node[above] {$\varepsilon$} (6)
            (4)     edge                                node[above] {$b$}           (5)
            (5)     edge                                node[above] {$\varepsilon$} (6)
            (6)     edge[bend right=90, looseness=1.5]  node[above] {$\varepsilon$} (1)
            (6)     edge                                node[above] {$\varepsilon$} (7)
            (7)     edge                                node[above] {$a$}           (8)
            (8)     edge                                node[above] {$\varepsilon$} (9)
            (8)     edge                                node[above] {$\varepsilon$} (11)
            (9)     edge                                node[above] {$a$}           (10)
            (10)    edge                                node[above] {$\varepsilon$} (13)
            (11)    edge                                node[above] {$b$}           (12)
            (12)    edge                                node[above] {$\varepsilon$} (13)
            ;
        \end{tikzpicture}
        \caption{识别 ${(a|b)}^{*} a {(a|b)}$ 的 NFA}
        \label{fig:NFA-2.12.a}
    \end{figure}
    然后使用算法 2.2 将其转换为 DFA\@:
    \begin{align}
         & \varepsilon\textrm{-}closure(0) = \left\{ 0, 1, 2, 4, 7\right\} = A    \\
         & \varepsilon\textrm{-}closure \left( move \left( A, a\right)\right)
        = \varepsilon\textrm{-}closure \left( \left\{ 3, 8\right\}\right)
        = \left\{ 1, 2, 3, 4, 6, 7, 8, 9, 11\right\} = B                          \\
         & \varepsilon\textrm{-}closure \left( move \left( A, b\right)\right)
        = \varepsilon\textrm{-}closure \left( \left\{ 5\right\}\right)
        = \left\{ 1, 2, 4, 5, 6, 7\right\} = C                                    \\
         & \varepsilon\textrm{-}closure \left( move \left( B, a\right)\right)
        = \varepsilon\textrm{-}closure \left( \left\{ 3, 8, 10\right\}\right)
        = \left\{ 1, 2, 3, 4, 6, 7, 8, 9, 10, 11, 13\right\} = D                  \\
         & \varepsilon\textrm{-}closure \left( move \left( B, b\right)\right)
        = \varepsilon\textrm{-}closure \left( \left\{ 5, 12\right\}\right)
        = \left\{ 1, 2, 4, 5, 6, 7, 12, 13\right\} = E                            \\
         & \varepsilon\textrm{-}closure \left( move \left( C, a\right)\right)
        = \varepsilon\textrm{-}closure \left( \left\{ 3, 8\right\}\right) = B     \\
         & \varepsilon\textrm{-}closure \left( move \left( C, b\right)\right)
        = \varepsilon\textrm{-}closure \left( \left\{ 5\right\}\right) = C        \\
         & \varepsilon\textrm{-}closure \left( move \left( D, a\right)\right)
        = \varepsilon\textrm{-}closure \left( \left\{ 3, 8, 10\right\}\right) = D \\
         & \varepsilon\textrm{-}closure \left( move \left( D, b\right)\right)
        = \varepsilon\textrm{-}closure \left( \left\{ 5, 12\right\}\right) = E    \\
         & \varepsilon\textrm{-}closure \left( move \left( E, a\right)\right)
        = \varepsilon\textrm{-}closure \left( \left\{ 3, 8\right\}\right) = B     \\
         & \varepsilon\textrm{-}closure \left( move \left( E, b\right)\right)
        = \varepsilon\textrm{-}closure \left( \left\{ 5\right\}\right) = C
    \end{align}
    $A$ 是开始状态, $D$ 和 $E$ 是接受状态, 完整的转换表 \textit{Dtran} 如下
    表~\ref{tab:Dtran-2.12.a} 所示, 构造出对应的 DFA 如图~\ref{fig:DFA-2.12.a}
    所示.
    \begin{table}[ht]
        \centering
        \begin{tabular}{c|cc}
            \hline
            \multicolumn{1}{c|}{\multirow{2}{*}{状态}} & \multicolumn{2}{c}{输入符号}       \\ \cline{2-3}
                                                       & \multicolumn{1}{c|}{$a$}     & $b$ \\ \hline
            $A$                                        & \multicolumn{1}{c|}{$B$}     & $C$ \\ \hline
            $B$                                        & \multicolumn{1}{c|}{$D$}     & $E$ \\ \hline
            $C$                                        & \multicolumn{1}{c|}{$B$}     & $C$ \\ \hline
            $D$                                        & \multicolumn{1}{c|}{$D$}     & $E$ \\ \hline
            $E$                                        & \multicolumn{1}{c|}{$B$}     & $C$ \\ \hline
        \end{tabular}
        \caption{DFA 的转换表 \textit{Dtran}}
        \label{tab:Dtran-2.12.a}
    \end{table}
    \begin{figure}[ht]
        \centering
        \begin{tikzpicture}[node distance=2cm]
            \node[state]                        (A) {$A$};
            \node[left of=A]                    (n) {};
            \node[state, right of=A]            (B) {$B$};
            \node[state, below of=A]            (C) {$C$};
            \node[state, accepting, right of=B] (D) {$D$};
            \node[state, accepting, below of=B] (E) {$E$};

            \draw
            (n) edge                node[above] {开始}  (A)
            (A) edge                node[above] {$a$}   (B)
            (A) edge                node[left]  {$b$}   (C)
            (B) edge                node[above] {$a$}   (D)
            (B) edge[bend right=20] node[left]  {$b$}   (E)
            (C) edge                node[above] {$a$}   (B)
            (C) edge[loop left]     node[left]  {$b$}   (C)
            (D) edge[loop right]    node[right] {$a$}   (D)
            (D) edge                node[above] {$b$}   (E)
            (E) edge[bend right=20] node[right] {$a$}   (B)
            (E) edge                node[above] {$b$}   (C)
            ;
        \end{tikzpicture}
        \caption{子集构造法得到的 DFA}
        \label{fig:DFA-2.12.a}
    \end{figure}

    最后使用算法 2.3 化简得到的 DFA\@. 划分过程如下
    \begin{equation}
        \left\{ A, B, C, D, E\right\}
        \begin{cases}
            \left\{ A, B, C\right\} &
            \begin{cases}
                \left\{ A, C\right\} \\
                \left\{ B\right\}
            \end{cases}      \\
            \left\{ D, E\right\}    &
            \begin{cases}
                \left\{ D\right\} \\
                \left\{ E\right\}
            \end{cases}
        \end{cases}
    \end{equation}
    选择 $A$ 作为 $\left\{ A, C\right\}$ 的代表, 其他状态不变, 可以得到最简自动
    机. 它的转换表如下表~\ref{tab:Dtran-2.12.a-sim}, 最简的 DFA 如下
    图~\ref{fig:DFA-2.12.a-sim}.
    \begin{table}[ht]
        \centering
        \begin{tabular}{c|cc}
            \hline
            \multicolumn{1}{c|}{\multirow{2}{*}{状态}} & \multicolumn{2}{c}{输入符号}       \\ \cline{2-3}
                                                       & \multicolumn{1}{c|}{$a$}     & $b$ \\ \hline
            $A$                                        & \multicolumn{1}{c|}{$B$}     & $A$ \\ \hline
            $B$                                        & \multicolumn{1}{c|}{$D$}     & $E$ \\ \hline
            $D$                                        & \multicolumn{1}{c|}{$D$}     & $E$ \\ \hline
            $E$                                        & \multicolumn{1}{c|}{$B$}     & $A$ \\ \hline
        \end{tabular}
        \caption{DFA 的转换表 \textit{Dtran}}
        \label{tab:Dtran-2.12.a-sim}
    \end{table}
    \begin{figure}[ht]
        \centering
        \begin{tikzpicture}[node distance=2cm]
            \node[state]                                (A) {$A$};
            \node[left of=A]                            (n) {};
            \node[state, right of=A]                    (B) {$B$};
            \node[state, accepting, above right of=B]   (D) {$D$};
            \node[state, accepting, below right of=B]   (E) {$E$};

            \draw
            (n) edge                node[above] {开始} (A)
            (A) edge                node[above] {$a$}  (B)
            (A) edge[loop above]    node[above] {$b$}  (A)
            (B) edge                node[above] {$a$}  (D)
            (B) edge[bend right=20] node[below] {$b$}  (E)
            (D) edge[loop right]    node[right] {$a$}  (D)
            (D) edge                node[right] {$b$}  (E)
            (E) edge[bend right=20] node[above] {$a$}  (B)
            (E) edge[bend left=40]  node[above] {$b$}  (A)
            ;
        \end{tikzpicture}
        \caption{最简的 DFA}
        \label{fig:DFA-2.12.a-sim}
    \end{figure}
\end{solution}

% \begin{figure}[ht]
%     \centering
%     \begin{tikzpicture}
%     \end{tikzpicture}
%     \caption{}
%     \label{fig:}
% \end{figure}

\end{document}
