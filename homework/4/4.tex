\documentclass[boxes]{homework}

% This is a slightly-more-than-minimal document that uses the homework class.
% See the README at http://git.io/vZWL0 for complete documentation.

\name{傅申 PB20000051}        % Replace (Your Name) with your name.
\term{2022 秋}     % Replace (Current Term) with the current term.
\course{编译原理和技术}    % Replace (Course Name) with the course name.
\hwnum{4}          % Replace (Number) with the number of the homework.
\hwname{作业}
\problemname{习题}
\solutionname{解:}
\problemchap{3}

% Load any other packages you need here.
\usepackage[
    a4paper,
    top = 1.84cm,
    bottom = 1.84cm,
    left = 1.91cm,
    right = 1.91cm,
    includeheadfoot
]{geometry}
\fancyfootoffset{0pt} % make fancyhdr work properly
\usepackage{ctex}
\usepackage{multirow}

\begin{document}
%%%% Problem 4 (b) %%%%
\problemnumber{4}
\begin{problem}
文法
\begin{equation}
    R \to R \text{ `|' } R \mid R R \mid R* \mid (R) \mid a \mid b
\end{equation}
产生字母表 $\{a, b\}$ 上所有不含 $\varepsilon$ 的正规式. 注意, 第一条竖线加了引
号, 它是正规式的或运算符, 而不是文法产生式右部各选择之间的分割符, 另外 $*$
在这是一个普通的终结符. 该文法是二义的.
\begin{parts}
    \setcounter{enumi}{1}
    \part\label{prob:4.b}
    为该文法写一个等价的非二义文法. 它给予算符 $*$, 连接和 $\mid$ 的优先级
    和结合性同 2.2 节中定义的一致.
\end{parts}
\end{problem}
\begin{solution}
    \ref{prob:4.b} 如下
    \begin{align}
        R & \to R \text{ `|' } S \mid S   \\
        S & \to ST \mid T                 \\
        T & \to T* \mid (R) \mid a \mid b
    \end{align}
\end{solution}

%%%% Problem 10 %%%%
\problemnumber{10}
\begin{problem}
构造下面文法的 LL\.(1) 分析表.
\begin{equation}
    \begin{aligned}
        D & \to TL                                \\
        T & \to \textbf{int} \mid \textbf{real}   \\
        L & \to \textbf{id } R                    \\
        R & \to \textbf{, id } R \mid \varepsilon
    \end{aligned}
\end{equation}
\end{problem}
\begin{solution}
    各个非终结符的开始符号集合和后继符号集合为
    \begin{equation}
        \begin{aligned}
            FIRST(D)  & = FIRST(T) = \{\textbf{int}, \textbf{real}\} \\
            FIRST(L)  & = \{\textbf{id}\}                            \\
            FIRST(R)  & = \{\textbf{,} , \varepsilon\}               \\
            FOLLOW(D) & = FOLLOW(L) = FOLLOW(R) = \{\$\}             \\
            FOLLOW(T) & = FIRST(L) = \{\textbf{id}\}                 \\
        \end{aligned}
    \end{equation}
    构造出 LL\.(1) 分析表如下
    \begin{table}[ht]
        \centering
        \begin{tabular}{c|ccccc}
            \hline
            \multirow{2}{*}{非终结符} & \multicolumn{5}{c}{输入符号}                                                                                                                                                                            \\ \cline{2-6}
                                      & \multicolumn{1}{c|}{\textbf{int}}         & \multicolumn{1}{c|}{\textbf{real}}         & \multicolumn{1}{c|}{\textbf{id}}         & \multicolumn{1}{c|}{\textbf{,}}               & \$                  \\ \hline
            $D$                       & \multicolumn{1}{c|}{$D \to TL$}           & \multicolumn{1}{c|}{$D \to TL$}            & \multicolumn{1}{c|}{}                    & \multicolumn{1}{c|}{}                         &                     \\ \hline
            $T$                       & \multicolumn{1}{c|}{$T \to \textbf{int}$} & \multicolumn{1}{c|}{$T \to \textbf{real}$} & \multicolumn{1}{c|}{}                    & \multicolumn{1}{c|}{}                         &                     \\ \hline
            $L$                       & \multicolumn{1}{c|}{}                     & \multicolumn{1}{c|}{}                      & \multicolumn{1}{c|}{$L \to \textbf{id}$} & \multicolumn{1}{c|}{}                         &                     \\ \hline
            $R$                       & \multicolumn{1}{c|}{}                     & \multicolumn{1}{c|}{}                      & \multicolumn{1}{c|}{}                    & \multicolumn{1}{c|}{$R \to \textbf{, id } R$} & $R \to \varepsilon$ \\ \hline
        \end{tabular}
    \end{table}
\end{solution}

%%%% Problem 12 %%%%
\problemnumber{12}
\begin{problem}
下面的文法是否为 LL\.(1) 文法, 说明理由.
\begin{equation}
    \begin{aligned}
        S & \to AB \mid PQx         \\
        A & \to xy                  \\
        B & \to bc                  \\
        P & \to dP \mid \varepsilon \\
        Q & \to aQ \mid \varepsilon
    \end{aligned}
\end{equation}
\end{problem}
\begin{solution}
    该文法不是 LL\.(1) 文法. 理由如下:

    因为 $x\in FIRST(A)$, 所以 $x\in FIRST(AB)$. 而对于 $PQx$, 因为
    $\varepsilon\in FIRST(P)$, $\varepsilon\in FIRST(Q)$, 所以 $x\in FIRST(PQ)$,
    因此有 $x\in FIRST(AB) \cap FIRST(PQx) \implies FIRST(AB) \cap FIRST(PQx)
        \neq \varnothing$, 故该文法不是 LL\.(1) 的.
\end{solution}

%%%% Problem 19 (b) %%%%
\problemnumber{19}
\begin{problem}
考虑下面的文法:
\begin{equation}
    \label{eq:19.1}
    \begin{aligned}
        E & \to E + T \mid T      \\
        T & \to T F \mid F        \\
        F & \to F * \mid a \mid b
    \end{aligned}
\end{equation}
\begin{parts}
    \setcounter{enumi}{1}
    \part\label{prob:19.b}
    构造 LALR 分析表.
\end{parts}
\end{problem}
\begin{solution}
    拓广文法如下
    \begin{equation}
        \begin{aligned}
            E' & \to E                 \\
            E  & \to E + T \mid T      \\
            T  & \to T F \mid F        \\
            F  & \to F * \mid a \mid b
        \end{aligned}
    \end{equation}
    首先构造 LR\.(1) 项目集规范族如下, 其中项目集标号前的箭头代表了它们在 DFA 之
    中的转换关系.
    \begin{equation}
        \begin{aligned}
            I_{0}:                      & E' \to \cdot E, \$                  & I_{0}\xrightarrow{T} I_{2}:                      & E \to T., + / \$                    \\
                                        & E \to \cdot E + T, + / \$           &                                                  & T \to T \cdot F, + / a / b / \$     \\
                                        & E \to \cdot T, + / \$               &                                                  & F \to \cdot F *, + / * / a / b / \$ \\
                                        & T \to \cdot T F, + / a / b / \$     &                                                  & F \to \cdot a, + / * / a / b / \$   \\
                                        & T \to \cdot F, + / a / b / \$       &                                                  & F \to \cdot b, + / * / a / b / \$   \\
                                        & F \to \cdot F *, + / * / a / b / \$ & I_{0}, I_{6}\xrightarrow{F} I_{3}:               & T \to F \cdot, + / a / b / \$       \\
                                        & F \to \cdot a, + / * / a / b / \$   &                                                  & F \to F \cdot *, + / * / a / b / \$ \\
                                        & F \to \cdot b, + / * / a / b / \$   & I_{0}, I_{2}, I_{6}, I_{9}\xrightarrow{a} I_{4}: & F \to a \cdot, + / * / a / b / \$   \\
            I_{0}\xrightarrow{E} I_{1}: & E' \to E \cdot, \$                  & I_{0}, I_{2}, I_{6}, I_{9}\xrightarrow{b} I_{5}: & F \to b \cdot, + / * / a / b / \$   \\
                                        & E \to E \cdot + T, + / \$
        \end{aligned}
    \end{equation}
    \begin{equation}
        \begin{aligned}
            I_{1}\xrightarrow{+} I_{6}:        & E \to E + \cdot T, + / \$           & I_{3}, I_{7}\xrightarrow{*} I_{8}: & F \to F* \cdot, + / * / a / b / \$    \\
                                               & T \to \cdot T F, + / a / b / \$     & I_{6}\xrightarrow{T} I_{9}:        & E \to E + T \cdot, + / * / a / b / \$ \\
                                               & T \to \cdot F, + / a / b / \$       &                                    & T \to T \cdot F, + / a / b / \$       \\
                                               & F \to \cdot F *, + / * / a / b / \$ &                                    & F \to \cdot F *, + / * / a / b / \$   \\
                                               & F \to \cdot a, + / * / a / b / \$   &                                    & F \to \cdot b, + / * / a / b / \$     \\
                                               & F \to \cdot a, + / * / a / b / \$   &                                    & F \to \cdot b, + / * / a / b / \$     \\
            I_{2}, I_{9}\xrightarrow{F} I_{7}: & T \to TF \cdot, + / a / b / \$                                                                                   \\
                                               & F \to F \cdot *, + / * / a / b / \$
        \end{aligned}
    \end{equation}
    可以看出 LR\.(1) 项目规范族中没有同心的项目集, 因此可直接构造 LALR 分析表.
    先对产生式编号如下
    \begin{equation}
        \begin{aligned}
            (1) & E \to E + T & \qquad  (5) & F \to F * \\
            (2) & E \to T     & \qquad  (6) & F \to a   \\
            (3) & T \to T F   & \qquad  (7) & F \to b   \\
            (4) & T \to F     & \qquad
        \end{aligned}
    \end{equation}
    构造出 LALR 分析表如下
    \begin{table}[ht]
        \centering
        \caption{由文法~\ref{eq:19.1} 构造的 LALR 分析表.}
        \label{tab:19.1}
        \begin{tabular}{c|ccccc|ccc}
            \hline
            \multirow{2}{*}{状态} & \multicolumn{5}{c|}{动作} & \multicolumn{3}{c}{转移}                                         \\ \cline{2-9}
                                  & $a$                       & $b$                      & +    & *    & \$    & $E$ & $T$ & $F$ \\ \hline
            0                     & $s4$                      & $s5$                     &      &      &       & 1   & 2   & 3   \\ \hline
            1                     &                           &                          & $s6$ &      & $acc$ &     &     &     \\ \hline
            2                     & $s4$                      & $s5$                     & $r2$ &      & $r2$  &     &     & 7   \\ \hline
            3                     & $r4$                      & $r4$                     & $r4$ & $s8$ & $r4$  &     &     &     \\ \hline
            4                     & $r6$                      & $r6$                     & $r6$ & $r6$ & $r6$  &     &     &     \\ \hline
            5                     & $r7$                      & $r7$                     & $r7$ & $r7$ & $r7$  &     &     &     \\ \hline
            6                     & $s4$                      & $s5$                     &      &      &       &     & 9   & 3   \\ \hline
            7                     & $r3$                      & $r3$                     & $r3$ & $s8$ & $r3$  &     &     &     \\ \hline
            8                     & $r5$                      & $r5$                     & $r5$ & $r5$ & $r5$  &     &     &     \\ \hline
            9                     & $s4$                      & $s5$                     & $r1$ &      & $r1$  &     &     & 7   \\ \hline
        \end{tabular}
    \end{table}
\end{solution}
\end{solution}

%%%% Problem 27 (b) %%%%
\problemnumber{27}
\begin{problem}
文法 $G$ 的产生式如下
\begin{equation}
    \begin{aligned}
        S & \to I \mid R            \\
        I & \to d \mid  Id          \\
        R & \to WpF                 \\
        W & \to Wd \mid \varepsilon \\
        F & \to Fd \mid d
    \end{aligned}
\end{equation}
\begin{parts}
    \setcounter{enumi}{1}
    \part\label{prob:27.b}
    该文法是 LR\.(1) 文法吗? 为什么?
\end{parts}
\end{problem}
\begin{solution}
    拓广文法为
    \begin{equation}
        \begin{aligned}
            S' & \to S                   \\
            S  & \to I \mid R            \\
            I  & \to d \mid  Id          \\
            R  & \to WpF                 \\
            W  & \to Wd \mid \varepsilon \\
            F  & \to Fd \mid d
        \end{aligned}
    \end{equation}
    可以构造出 $I_{0}$ 如下
    \begin{equation}
        \begin{aligned}
            S' & \to \cdot S, \$       \\
            S  & \to \cdot I, \$       \\
            S  & \to \cdot R, \$       \\
            I  & \to \cdot d, d / \$   \\
            I  & \to \cdot I d, d / \$ \\
            R  & \to \cdot W p F, \$   \\
            W  & \to \cdot WpF, p / d  \\
            W  & \to \cdot , p / d     \\
        \end{aligned}
    \end{equation}
    只关注这两个项目: $[I \to \cdot d, d]$ 和 $[W \to \cdot, d]$. 在分析器还未读
    入符号, 面临 $d$ 的情况下, 前者表明分析器可以移进 $d$; 后者表明分析器可以按
    $W\to \varepsilon$ 进行规约. 因此出现了移进--规约冲突, 所以文法不是 LR\.(1)
    的.
\end{solution}


\end{document}
