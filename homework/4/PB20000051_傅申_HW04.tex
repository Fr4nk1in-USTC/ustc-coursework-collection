\documentclass[11pt]{homework}

% TODO: replace these with your information
\newcommand{\hwname}{傅申}
\newcommand{\hwid}{PB20000051}
\newcommand{\hwtype}{计算机组成原理作业}
\newcommand{\hwnum}{4}

\begin{document}
\maketitle
% Your content
\question
$$
78.75 = (-1)^0 \times 1.00111011_2\times 2^{(133-127)}
$$
所以 78.75 的二进制表示为 \texttt{01000010100111011000000000000000}, 十六进制形式为 \texttt{0x429D8000}.
\question
\begin{arabicparts}
    \questionpart $27\% + 12 \% = 39\%$
    \questionpart 100\%
    \questionpart $26\% + 27\% + 12\% + 10\% + 2\% = 77\%$
\end{arabicparts}
\question
当前指令为 \texttt{0000000\_01100\_01101\_011\_10100\_0100011}, 即 \texttt{sd x12, 20(x13)}.
\begin{arabicparts}
    \questionpart ALUop 为 \texttt{00}, ALU 控制线为 \texttt{0010}.
    \questionpart 新 PC 地址为 PC + 4, 即 \texttt{0xbfc00390}. 计算出该 PC 值的通路的序号为 $1\to 3\to 12\to 1$.
    \questionpart ALU 的输入数值为 \texttt{Reg[x13]} 和 \texttt{0x00000014}. \\
    3 号加法器的输入数值为 \texttt{0xBFC0038C} 和 \texttt{0x00000004}. \\
    11 号加法器的输入数值为 \texttt{0xBFC0038C} 和 \texttt{0x00000028}.
\end{arabicparts}
\question
\begin{arabicparts}
    \questionpart $40+235+160+45+230+45+15=770$ ps
    \questionpart $40+235+160+45+230+235+45+15=1005$ ps
    \questionpart $40+235+160+45+230+235=945$ ps
    \questionpart $40+235+160+45+230+10+45+15=780$ ps
    \questionpart $40+235+160+45+230+45+15=770$ ps
    \questionpart 1005 ps
\end{arabicparts}
\end{document}
