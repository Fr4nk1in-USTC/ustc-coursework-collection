\documentclass[11pt]{homework}

% TODO: replace these with your information
\newcommand{\hwname}{傅申}
\newcommand{\hwid}{PB20000051}
\newcommand{\hwtype}{操作系统作业}
\newcommand{\hwnum}{4}

\usepackage{calc}
\usepackage{enumitem}
\begin{document}
\setlength{\intextsep}{0em}
\setlength{\belowdisplayskip}{.25em} \setlength{\belowdisplayshortskip}{.25em}
\setlength{\abovedisplayskip}{.25em} \setlength{\abovedisplayshortskip}{.25em}
\maketitle
% Your content
\question
\begin{description}[leftmargin = !, labelwidth = \widthof{\bfseries Segmentation Fault}]
    \item[Segmentation Fault] When you access a memory that is not allowed (e.g., write to read-only
        segments and access unused or unallocated segments), the OS returns you segmentation fault.
    \item[TLB] The TLB is a special, small, fast-lookup hardware called translation look-aside buffer.
        It is associative, high-speed memory. It is used to boost the performance of page table.
    \item[Page Fault] If a page marked invalid is accessed, the CPU generates an interrupt called
        page fault, causing the page fault handling routine to be called.
    \item[Demand Paging] Demand paging is a technique that pages are loaded only as they are needed.
\end{description}
\question
Thrashing refers to an instance of high paging activity. A process is thrashing means it is spending
more time paging than executing. It results in severe performance problem.

When the total demand for frames is greater than the total number of available frames, thrashing will
occur, because some processes will not have enough frames. Thus, the process will replace pages that
will be needed again right away, causing frequent page faults.
\question
\begin{alphaparts}
    \questionpart It takes $50 + 50 = 100$ nanoseconds.
    \questionpart The effective memory reference time is $75\%\times (2 + 50) + 25\% \times 100 = 64$
    nanoseconds.
\end{alphaparts}
\question
\begin{description}[leftmargin = !, labelwidth = \widthof{\bfseries TLB miss with no page fault}]
    \item[TLB miss with no page fault] The page is in the memory but the page table entry is not
        cached in the TLB.
    \item[TLB miss and page fault] The page is not in the memory and of course the page table entry
        is not cached in the TLB.
    \item[TLB hit and no page fault] The page is in the memory and the page table entry is cached in
        the TLB.
    \item[TLB hit and page fault] It cannot occur, since if the page table entry is cached in the
        TLB, the page must be in the memory, causing no page fault.
\end{description}
\question
Assume the page fault rate is $p$, the effective access time is
$$
    (1 - p) \times 100 \mathrm{ns} + p \times (30\% \times 8 \mathrm{ms} + 70\% \times 20 \mathrm{ms})
    = (100 + 16399900p) \mathrm{ns} \le 200 \mathrm{ns}
$$
Which means the maximum acceptable page-fault rate is
$$
    \max p = \frac{1}{163999} \approx 6.098\times 10^{-6}
$$

\question
The execution of each algorithm is shown in the tables below. The cell with a `-' represents
an empty page. And the cell with a boxed number represents a page replacement, i.e., a page fault.
\begin{table}[!htbp]
    \centering
    \caption{LRU algorithm}
    \label{table:lru}
    \resizebox{\textwidth}{!}{%
        \begin{tabular}{c|cccccccccccccccccccc}
            Page reference & 7        & 2        & 3        & 1        & 2 & 5        & 3        & 4        & 6        & 7        & 7 & 1        & 0        & 5        & 4        & 6        & 2        & 3        & 0        & 1        \\ \hline
            Page \#1       & \fbox{7} & 7        & 7        & \fbox{1} & 1 & 1        & \fbox{3} & 3        & 3        & \fbox{7} & 7 & 7        & 7        & \fbox{5} & 5        & 5        & \fbox{2} & 2        & 2        & \fbox{1} \\
            Page \#2       & -        & \fbox{2} & 2        & 2        & 2 & 2        & 2        & \fbox{4} & 4        & 4        & 4 & \fbox{1} & 1        & 1        & \fbox{4} & 4        & 4        & \fbox{3} & 3        & 3        \\
            Page \#3       & -        & -        & \fbox{3} & 3        & 3 & \fbox{5} & 5        & 5        & \fbox{6} & 6        & 6 & 6        & \fbox{0} & 0        & 0        & \fbox{6} & 6        & 6        & \fbox{0} & 0        \\
        \end{tabular}%
    }

\end{table}
\begin{table}[!htbp]
    \centering
    \caption{FIFO algorithm}
    \label{table:fifo}
    \resizebox{\textwidth}{!}{%
        \begin{tabular}{c|cccccccccccccccccccc}
            Page reference & 7        & 2        & 3        & 1        & 2 & 5        & 3 & 4        & 6        & 7        & 7 & 1        & 0        & 5        & 4        & 6        & 2        & 3        & 0        & 1        \\ \hline
            Page \#1       & \fbox{7} & 7        & 7        & \fbox{1} & 1 & 1        & 1 & 1        & \fbox{6} & 6        & 6 & 6        & \fbox{0} & 0        & 0        & \fbox{6} & 6        & 6        & \fbox{0} & 0        \\
            Page \#2       & -        & \fbox{2} & 2        & 2        & 2 & \fbox{5} & 5 & 5        & 5        & \fbox{7} & 7 & 7        & 7        & \fbox{5} & 5        & 5        & \fbox{2} & 2        & 2        & \fbox{1} \\
            Page \#3       & -        & -        & \fbox{3} & 3        & 3 & 3        & 3 & \fbox{4} & 4        & 4        & 4 & \fbox{1} & 1        & 1        & \fbox{4} & 4        & 4        & \fbox{3} & 3        & 3        \\
        \end{tabular}
    }
\end{table}
\begin{table}[!htbp]
    \centering
    \caption{Optimal algorithm}
    \label{table:opt}
    \resizebox{\textwidth}{!}{%
        \begin{tabular}{c|cccccccccccccccccccc}
            Page reference & 7        & 2        & 3        & 1        & 2 & 5        & 3 & 4        & 6        & 7        & 7 & 1 & 0        & 5 & 4        & 6        & 2        & 3        & 0 & 1 \\ \hline
            Page \#1       & \fbox{7} & 7        & 7        & \fbox{1} & 1 & 1        & 1 & 1        & 1        & 1        & 1 & 1 & 1        & 1 & 1        & 1        & 1        & 1        & 1 & 1 \\
            Page \#2       & -        & \fbox{2} & 2        & 2        & 2 & \fbox{5} & 5 & 5        & 5        & 5        & 5 & 5 & 5        & 5 & \fbox{4} & \fbox{6} & \fbox{2} & \fbox{3} & 3 & 3 \\
            Page \#3       & -        & -        & \fbox{3} & 3        & 3 & 3        & 3 & \fbox{4} & \fbox{6} & \fbox{7} & 7 & 7 & \fbox{0} & 0 & 0        & 0        & 0        & 0        & 0 & 0 \\
        \end{tabular}
    }
\end{table}

The number of page fault is:
\begin{description}[leftmargin = !, labelwidth = \widthof{\bfseries Optimal}, topsep = 0pt, itemsep = 0pt, parsep = 0pt]
    \item[LRU] 18 page faults.
    \item[FIFO] 17 page faults.
    \item[Optimal] 13 page faults.
\end{description}
\question
The Belady's anomaly is: for some page-replacement algorithms, the page fault rate may increase
as the number of frames increases.

The feature of stack algorithms which never exhibit is: the set of pages in memory for $n$ frames is
always a subset of the set of pages in memory for $n + 1$ frames.
\end{document}
