\documentclass[boxes]{homework}

% This is a slightly-more-than-minimal document that uses the homework class.
% See the README at http://git.io/vZWL0 for complete documentation.

\name{傅申 PB20000051}        % Replace (Your Name) with your name.
\term{2022 秋}     % Replace (Current Term) with the current term.
\course{算法基础}    % Replace (Course Name) with the course name.
\hwnum{5}          % Replace (Number) with the number of the homework.
\hwname{作业}
\problemname{}
\solutionname{解:}

% Load any other packages you need here.
\usepackage[
    a4paper,
    top = 2.54cm,
    bottom = 2.54cm,
    left = 1.91cm,
    right = 1.91cm,
    includeheadfoot
]{geometry}
\fancyfootoffset{0pt} % make fancyhdr work properly
\usepackage{ctex}

\begin{document}
%%%% Problem 15.1-2 %%%%
\problemchap{15}
\problempart{1}
\problemnumber{2}
\begin{problem}
举反例证明下面的 ``贪心'' 策略不能保证总是得到最优切割方案. 定义长度为 $i$ 的钢
条的\textbf{密度}为 $p_{i} / i$, 即每英寸的价值. 贪心策略将长度为 $n$ 的钢条切割
下长度为 $i$ $(1\leqslant i\leqslant n)$ 的一段, 其密度最高. 接下来继续使用相同
的策略切割长度为 $n - i$ 的剩余部分.
\end{problem}
\begin{solution}
    假设 $n = 5$, 长度为 1 \textasciitilde{} 5 的钢条的密度依次为: 1, 2, 2, 8,
    7. 则按照贪心策略会将钢条切割为长度为 1 和 4 的两部分, 总价值为 $1\times 1 +
        4 \times 8 = 33$; 而最优的切割方案为不切割, 总价值为 $5 \times 7 = 35$. 所以
    贪心策略不能保证总是得到最优的切割方案.
\end{solution}

%%%% Problem 15.2-3 %%%%
\problempart{2}
\problemnumber{3}
\begin{problem}
用代入法证明递归公式 \hyperref[eq:15.2.3.1]{(15.6)} 的结果为 $\Omega(2^{n})$.
\begin{equation}
    \label{eq:15.2.3.1}
    P(n) = \begin{cases}
        1                                               & n = 1         \\
        \displaystyle \sum_{k = 1}^{n - 1} P(k)P(n - k) & n \geqslant 2
    \end{cases}
    \tag{15.6}
\end{equation}
\end{problem}
\begin{solution}
    恰当选择常数 $c \geqslant 1$, 有 $P(n)\geqslant c \cdot 2^{n}$. 假定此上界对
    所有正数 $m < n$ 都成立, 将此代入递归式, 有
    \begin{equation}
        \begin{aligned}
            P(n)
             & = \sum_{k = 1}^{n - 1} P(k)P(n - k) \\
             & \geqslant c^{2} (n - 1) 2^{n}       \\
             & \geqslant c \cdot 2^{n}
        \end{aligned}
    \end{equation}
    所以递归公式的结果为 $\Omega(2^{n})$.
\end{solution}

%%%% Problem 15.2-4 %%%%
\problempart{2}
\problemnumber{4}
\begin{problem}
对输入链长度为 $n$ 的矩阵链乘问题, 描述其字问题图: 它包含多少个顶点? 包含多少条
边? 这些边分别连接哪些顶点?
\end{problem}
\begin{solution}
    字问题图的顶点都能用 $v_{i, j}$ ($i \leqslant j$) 表示, 共 $n(n + 1)/2$ 个,
    其中
    \begin{itemize}
        \item 若 $i = j$, 则顶点 $v_{i, j}$ 没有任何出边.
        \item 若 $i < j$, 则对所有的 $i \leqslant k < j$, $v_{i, j}$ 都有两条出
              边分别指向 $v_{i, k}$ 和 $v_{k + 1, j}$.
    \end{itemize}
    则边数为
    \begin{equation}
        \sum_{i = 1}^{n}\sum_{j = i + 1}^{n} 2 (j - i)
        = \sum_{i = 1}^{n} (n - i)(n - i + 1)
        = \sum_{i = 1}^{n} \left( i^{2} - i\right)
        = \frac{ n(n+1)(2n+1) }{ 6 } - \frac{ n(n+1) }{ 2 }
        = \frac{ (n - 1)n(n + 1) }{ 3 }
    \end{equation}
\end{solution}
\end{document}
